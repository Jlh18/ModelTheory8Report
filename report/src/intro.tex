It is a basic fact of linear algebra that any linear map
between vector spaces of the same finite dimension is
injective if and only if it is surjective.
\linkto{ax_groth_thm}{Ax-Grothendieck} says that this is
partly true for polynomial maps.

Here are some examples of polynomial maps
\begin{itemize}
  \item Surjective but not injective: $f : \C \to \C := x \mapsto x^{2}$
  \item Neither surjective nor injective:
      $f : \C^{2} \to \C^{2} := (x,y) \mapsto (x,xy)$
  \item Bijective:
      $f : \C^{3} \to \C^{3} := (x,y,z) \mapsto ( x , y , z + xy )$
\end{itemize}

One might try very hard to look for an example of an injective polynomial map
that is not surjective.
Replacing $\C$ with an arbitrary field, we notice that
surjectivity and injectivity coincide on finite fields.
% and indeed on small enough fields (locally finite fields).
% Then we notice that this implies they coincide for
Ax-Grothendieck states that over any algebraically closed field,
injectivity of a polynomial map implies surjectivity.
