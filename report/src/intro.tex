\subsection*{Motivation}

It is a basic fact of linear algebra that any linear map
between vector spaces of the same finite dimension is
injective if and only if it is surjective.
\linkto{ax_groth_thm}{Ax-Grothendieck} says that this is
partly true for \linkto{dfn_poly_map}{polynomial maps}.

Here are some examples of polynomial maps
\begin{itemize}
  \item Surjective but not injective: $f : \C \to \C := x \mapsto x^{2}$
  \item Neither surjective nor injective:
      $f : \C^{2} \to \C^{2} := (x,y) \mapsto (x,xy)$
  \item Bijective:
      $f : \C^{3} \to \C^{3} := (x,y,z) \mapsto ( x , y , z + xy )$
\end{itemize}

One might try very hard to look for an example of an injective polynomial map
that is not surjective.
Replacing $\C$ with a finite field, we notice that
surjectivity and injectivity coincide in this case.
Ax-Grothendieck states that over any algebraically closed field,
injectivity of a polynomial map implies surjectivity,
and the proof we will give is model theoretic,
roughly saying ``we may reduce to the finite field case''.

This project formalizes this proof of Ax-Grothendieck in
\href{https://leanprover.github.io/}{\code{lean}},
which is a theorem prover\footnote{
  As far as I am aware this is the first attempt to formalize Ax-Grothendieck in
  a theorem prover. }.
My github repository for this project is accessible at
\url{github.com/Jlh18/ModelTheoryInLean8}.

\subsection*{References and acknowledgments}

The standard library for mathematics in \code{lean} is a community project called
\code{mathlib},
which contains results about rings, polynomials, algebraic closures and so on.
Currently \code{mathlib} is developing its model theory library,
but this \textit{is not} what I am using.
My \code{lean} code extends both \code{mathlib} and an existing model theory library
called the \code{flypitch} project \cite{flypitch},
which built the framework for basic model theory and proved the
independence of the continuum hypothesis.
As \code{mathlib} is frequently updated,
old projects need maintenance in order to be compatible with modern versions.
I have Yakov Pechersky and Kevin Buzzard to thank for laboriously helping me
update the \code{flypitch} project to be compatible with newer versions of \code{mathlib}.

My original work all lies in the folder \code{src/Rings},
even though a lot of the code (such as the proof of \linkto{vaught_test}{Vaught's test})
is general model theory.
Most of the content of \code{src/Rings} is my original work,
except some contributions from Chris Hughes,
which I specify in comments above the theorems and definitions.

Most definitions and proofs in this document follow the ideas given in
David Marker's book on Model Theory \cite{marker} and formalization of the theory in \code{flypitch}.

\subsection*{Overview}



\subsection*{Formalization}

% ? The choice of generality in which to work
% (model theory: multi-sorted, algebraic/relational.
% Maybe mention previous attempt at non-standard analysis being seperate from other bits of mathlib)

% ? Obvious identifications and transferring structure + properties across identified objects
% ? (model theory: ring theory models and actual actual rings) %? important from a computer science POV

% ? Cardinality is a pain in the ass

% ? Finite things e.g. sums

% ? Structure with missing properties, e.g. addition and multiplication for ring terms.
