It is a basic fact of linear algebra that any linear map
between vector spaces of the same finite dimension is
injective if and only if it is surjective.
\linkto{ax_groth_thm}{Ax-Grothendieck} says that this is
partly true for \linkto{dfn_poly_map}{polynomial maps}.

Here are some examples of polynomial maps
\begin{itemize}
  \item Surjective but not injective: $f : \C \to \C := x \mapsto x^{2}$
  \item Neither surjective nor injective:
      $f : \C^{2} \to \C^{2} := (x,y) \mapsto (x,xy)$
  \item Bijective:
      $f : \C^{3} \to \C^{3} := (x,y,z) \mapsto ( x , y , z + xy )$
\end{itemize}

One might try very hard to look for an example of an injective polynomial map
that is not surjective.
Replacing $\C$ with a finite field, we notice that
surjectivity and injectivity coincide in this case.
Ax-Grothendieck states that over any algebraically closed field,
injectivity of a polynomial map implies surjectivity,
and the proof we will give is model theoretic,
roughly saying ``we may reduce to the finite field case''.

This project formalizes this proof of Ax-Grothendieck in \texttt{lean}\footnote{
  As far as I am aware this is the first attempt to formalize Ax-Grothendieck in
  proof verification software. }.
My github repository for this project is accessible at
\url{github.com/Jlh18/ModelTheoryInLean8}.

I work on a fork of the \texttt{flypitch} project \cite{flypitch},
which built the framework for basic model theory and proved the
independence of the continuum hypothesis.
I have made many updates to this fork to make it compatible with
more modern versions of \texttt{mathlib}.
My original work all lies in the folder \texttt{src/Rings},
even though a lot of the code (such as the proof of \linkto{vaught_test}{Vaught's test})
is general model theory.
