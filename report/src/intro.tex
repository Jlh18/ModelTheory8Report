\subsection*{Motivation}

It is a basic fact of linear algebra that any linear map
between vector spaces of the same finite dimension is
injective if and only if it is surjective.
\linkto{ax_groth_thm}{Ax-Grothendieck} says that this is
partly true for \linkto{dfn_poly_map}{polynomial maps}.

Here are some examples of polynomial maps
\begin{itemize}
  \item Surjective but not injective: $f : \C \to \C := x \mapsto x^{2}$
  \item Neither surjective nor injective:
      $f : \C^{2} \to \C^{2} := (x,y) \mapsto (x,xy)$
  \item Bijective:
      $f : \C^{3} \to \C^{3} := (x,y,z) \mapsto ( x , y , z + xy )$
\end{itemize}

One might try very hard to look for an example of an injective polynomial map
that is not surjective.
Replacing $\C$ with a finite field, we notice that
surjectivity and injectivity coincide in this case.
Ax-Grothendieck states that over any algebraically closed field,
injectivity of a polynomial map implies surjectivity,
and the proof we will give is model theoretic,
roughly saying ``we may reduce to the finite field case''.

This project formalizes this proof of Ax-Grothendieck in
\href{https://leanprover.github.io/}{\code{lean}},
which is a theorem prover\footnote{
  As far as I am aware this is the first attempt to formalize Ax-Grothendieck in
  a theorem prover. }.
My github repository for this project is accessible at
\url{github.com/Jlh18/ModelTheoryInLean8}.

\subsection*{References and acknowledgments}

The standard library for mathematics in \code{lean} is a community project called
\code{mathlib},
which contains results about rings, polynomials, algebraic closures and so on.
Currently \code{mathlib} is developing its model theory library,
but this \textit{is not} what I am using.
My \code{lean} code extends both \code{mathlib} and an existing model theory library
called the \code{flypitch} project \cite{flypitch},
which built the framework for basic model theory and proved the
independence of the continuum hypothesis.
As \code{mathlib} is frequently updated,
old projects need maintenance in order to be compatible with modern versions.
I have Yakov Pechersky and Kevin Buzzard to thank for laboriously helping me
update the \code{flypitch} project to be compatible with newer versions of \code{mathlib}.

My original work all lies in the folder \code{src/Rings},
even though a lot of the code (such as the proof of \linkto{vaught_test}{Vaught's test})
is general model theory.
Most of the content of \code{src/Rings} is my original work,
except some contributions from Chris Hughes,
which I specify in comments above the theorems and definitions.

Most definitions and proofs in this document follow the ideas given in
David Marker's book on Model Theory \cite{marker}
and formalization of the theory in \code{flypitch}.

\subsection*{Summary of contents}

Throughout this document I try to incorporate explanations of \code{lean} code,
including some type theoretic notions.
With the intention of not distracting from the project,
I have incorporated this alongside my exposition of the theory;
most of these type theoretic concepts are explained in
\linkto{model_theory_background}{section 2}.

In order to be true to the formalization process
I tend to work backwards:
I first prove the main result from its lemmas,
then prove the lemmas in later sections.
Similarly, whilst formalizing a result I would first state the lemma in \code{lean}
and try to establish whether the lemma is the appropriate approach
to the final result before (removing the placeholder \code{sorry} and) providing a proof.
Furthermore, model theoretic lemmas tend to be technical and better motivated using results.

In \linkto{model_theory_background}{section 2} I introduce the basics of model theory and
the model theory in the language of rings,
motivated by the aim of writing down the sentence
``any degree two polynomial over a given ring has a root'' as a mathematical object.
This then allows us to classify the ``theory of rings'', ``theory of fields'' and
``theory of algebraically closed fields'' ($\ACF$), also as mathematical objects.

In \linkto{algebraic_objects_iff_models}{section 3}
I show that models of the theory of rings are the same thing
as ``actual'' rings, where ``actual'' rings are the rings defined in \code{mathlib}.
Similarly, I show that models of the theory of fields are the same thing as ``actual'' fields.
These facts are usually glossed over in the literature,
since it is obvious that the finite collection of sentences in the theories of
rings and fields correspond exactly to the ``actual'' properties they must satisfy.
However, it is more complicated to do the same for $\ACF$,
since ``actual'' algebraically closed fields are defined by
quantifying over all polynomials on the field,
whereas model theoretically we need a trick in order to
do the same quantification (we must write each possible polynomial out).

We informally call this kind of conversion between model theory and ``actual'' algebra
``internal completeness and soundness''.
From a computer science perspective,
internal completeness and soundness is important as a user interface for
converting between a model theory library and \code{mathlib}.
It is essential for the purpose of making model theoretic results that
are usable in formalization of other areas of mathematics.
In this sense this project not only formalizes the model theory,
but also shows that model theory can be a practical tool for formalizing
purely algebraic results.

In the first half of \linkto{section_4}{section 4} I introduce Ax-Grothendieck,
provide an overview of its proof,
discuss its model theoretic statement,
then show the relevant internal completeness and soundness lemmas.
In the latter half I state without proof \linkto{lefschetz}{the Lefschetz principle},
and prove Ax-Grothendieck using it.
The rest of the document is dedicated to proving Lefschetz.

An important difficulty with formalizing the model theoretic statement of Ax-Grothendieck
is simply writing down long and complicated terms and formulas.
For example, we will need terms that involve sums indexed by some set;
see the \linkto{ax_groth_formula}{Ax-Grothendieck formula}.

\[ \sum_{f\texttt{ : monom\_deg\_le n d}} p_{f}\prod_{i < n} x^{f i}\]

However, the collection of terms (with addition as applying the function symbol $+$)
does not form a ring (addition of terms in the language of rings is not even associative).
This makes writing down a sum indexed by a set very difficult;
a problem I circumvented by using sums indexed by a list.

In \linkto{lefschetz_proof}{section 5} I prove Lefschetz,
which requires two lemmas: the compactness theorem (which we do not prove)
and \linkto{vaught_test}{Vaught's test}.

In \linkto{section_6}{section 6} I prove Vaught's test,
which is a corollary of the lemma \linkto{upwards_lowenheim_skolem}{Upwards L\"{o}wenheim-Skolem}.

In \linkto{section_7}{section 7} I prove Upwards L\"{o}wenheim-Skolem.
This is a result about the cardinality of models of a theory,
and its proof relies on two different kind of constructions.
The first is constructing new languages and theories that
extend our original language and theory.
The second is constructing a model associated to a theory (satisfying some properties)
called \linkto{term_model}{\code{term\_model}}.
We then work on computing the cardinality of this model,
which requires some general lemmas relating the cardinalities
of function symbols, terms, and formulas in a language,
which we prove in \linkto{cardinality_lemmas}{section 8}.

The cardinality of the collection of terms or formulas in a language
is often not discussed in detail,
though it is very important in many model theory constructions.
A potential set-theoretic formalization of this has cardinality baked into the framework:
terms and formulas can be defined as strings of symbols that satisfy certain properties
(from which we extract their induction principles).
Thus terms and formulas can be injected into the collection of
all strings of symbols, which has an obvious cardinality.
My proof of these cardinality bounds for terms and formulas
follows exactly this idea.\footnote{
  That is not to say this formalization is advantageous:
  the way terms and formulas are formalized in \code{flypitch}
  means that induction on terms and formulas is baked into their definition,
  which I believe is much more important than their cardinality.
  Inductive principles do not come for free with strings of symbols.}

