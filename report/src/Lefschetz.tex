Returning to model theory of algebraically closed fields.
We begin by introducing the notion of a complete theory:

\begin{dfn}[Complete theories]
    \link{dfn_complete_theory}
    An $L$-theory $T$ is \textit{complete}
    when either of the following equivalent definitions hold:
    \begin{itemize}
      \item  $T$ deduces any sentence of its negation
    \begin{lstlisting}
      def is_complete' (T : Theory L) : Prop :=
      ∀ (ϕ : sentence L), T ⊨ ϕ ∨ T ⊨ ∼ ϕ \end{lstlisting}
      \item Sentences true in any model are deduced by the theory.
    \begin{lstlisting}
      def is_complete'' (T : Theory L) : Prop :=
      ∀ (M : Structure L) (hM : nonempty M) (ϕ : sentence L), M ⊨ T → M ⊨ ϕ → T ⊨ ϕ \end{lstlisting}
      \item All models of $T$ satisfy the same sentences
            (``are elementarily equivalent'').
    \end{itemize}
    Note that the definition \texttt{is\_complete} from the flypitch project
    is stronger than these conditions, and is useful when constructing
    theories with nice properties\footnote{
      Personally, I prefer the word maximal consistent theory for
      their definition \texttt{is\_complete}}.
    However in practice there is no reason to throw that many sentences
    into our language, so we use the versions above.
\end{dfn}
\begin{proof}
  The statement is
\begin{lstlisting}
  lemma is_complete''_iff_is_complete' {T : Theory L} :
    is_complete' T ↔ is_complete'' T := sorry \end{lstlisting}
  The forward direction involves casing on the hypothesis of $T \vDash \phi$
  or $T \vDash \neg \phi$, in the first case we are done,
  and in the second we get a contradiction by
  $\phi$ being both true and false in our model $M$.
\begin{lstlisting}
  { intros H M hM ϕ hMT hMϕ,
    cases H ϕ with hTϕ hTϕ,
    { exact hTϕ },
    {
      have hbot := hTϕ hM hMT,
      rw realize_sentence_not at hbot,
      exfalso,
      exact hbot hMϕ } },
\end{lstlisting}
    On the other hand we need to case on whether $T$
    is consistent or not.
    When $T$ is consistent we can show $T$ deduces
    $\phi$ or its negation by checking in that model,
    otherwise $T$ should deduce anything.
\begin{lstlisting}
  { intros H ϕ,
    by_cases hM : ∃ M : Structure L, nonempty M ∧ M ⊨ T,
    {
      rcases hM with ⟨ M , hM0 , hMT ⟩,
      by_cases hMϕ : M ⊨ ϕ,
      { left, exact H M hM0 ϕ hMT hMϕ },
      {
        right,
        rw ← realize_sentence_not at hMϕ,
        exact H M hM0 _ hMT hMϕ} },
    { left,
      intros M hM0 hMT,
      exfalso,
      apply hM ⟨ M , hM0 , hMT ⟩} } \end{lstlisting}
\end{proof}

\begin{prop}[Lefschetz principle]
    \link{lefschetz}
    Let $\phi$ be a sentence in the language of rings.
    Then the following are equivalent:
    \begin{enumerate}
        \item Some model of $\ACF_0$ satisfies $\phi$.
        (If you like $\C \vDash \phi$.)
        \item $\ACF_0 \vDash \phi$
        \item There exists $n \in \N$ such that for any prime $p$
            greater than $n$, $\ACF_p \vDash \phi$
        \item There exists $n \in \N$ such that for any prime $p$
        greater than $n$, some model of $\ACF_p$ satisfies $\phi$.
    \end{enumerate}
    The first and last equivalences are due to the theories $\ACF_{p}$
    being complete for any $p$ ($0$ or prime).
\end{prop}

To prove the above we need the following
\begin{itemize}
  \item \link{vaught_test}{Vaught's test} for showing a theory is complete
        (this does the first and last equivalences and
        is needed in the middle equivalence)
  \item The compactness theorem for the middle equivalence.
\end{itemize}
In this section we will introduce these notions properly and how
they are used.
Vaught's test will be proven in a \linkto{vaught_proof}{later section}.
The compactness theorem will not be proven
(it was part of the flypitch project).

\subsection{Vaught's test}
Another way of expressing that a theory $T$ is complete is to
ask for models of $T$ to satisfy the same sentences
(that they are elementarily equivalent).
In particular it is known that isomorphic models %? Ref?
satisfy the same sentences.

\begin{dfn}[Categoricity]
    Given a language $L$ and a cardinal $\ka$,
    an $L$-theory $T$ is called $\ka$-categorical
    if any two models of $T$ of size $\ka$ are isomorphic.

    \begin{lstlisting}
  def categorical (κ : cardinal) (T : Theory L) :=
  ∀ (M N : Structure L) (hM : M ⊨ T) (hN : N ⊨ T), #M = κ → #N = κ → nonempty (M ≃[L] N) \end{lstlisting}
\end{dfn}

Vaught's test says that categoricity is a useful condition for showing a theory is complete.
Another condition we will need is that there are only infinite models to the theory

\begin{lstlisting}
  def only_infinite (T : Theory L) : Prop := ∀ (M : Model T), infinite M.1\end{lstlisting}

\begin{prop}[Vaught's Test]
  \link{vaught_test}
  Let $L$ be a language and $T$ be a consistent theory in the language $L$
  with only infinite models, such that it is $\ka$-categorical
  for some large enough cardinal $\ka$ (see below for details).
  Then $T$ is a complete theory.

  \begin{lstlisting}
  lemma is_complete'_of_only_infinite_of_categorical
    [is_algebraic L] {T : Theory L} (M : Structure L) (hM : M ⊨ T)
    (hinf : only_infinite T) {κ : cardinal}
    (hκ : ∀ n, #(L.functions n) ≤ κ) (hωκ : ω ≤ κ) (hcat : categorical κ T) :
    is_complete' T := sorry
\end{lstlisting}
  This may differ slightly to the statement in other sources;
  the reason for the choice of these (stronger than usual)
  hypotheses will be discussed
  in the \linkto{vaught_proof}{section dedicated it its proof}.
  % In practice it is less work to prove this when $L$ has no relation symbols,
  % which is the case we are interested in (we say $L$ \textit{is algebraic}).
  % Another practical simplification is asking for $\kappa$ to be larger than
  % the collection of all function symbols.
  % The proof that Marker gives %? Ref
  % only requires that $\kappa$ is larger than the collection of constant symbols.
\end{prop}

We apply Vaught's test in our case to show that the theory
of algebraically closed fields of a fixed characteristic is complete.

\begin{prop}
  $\ACF_{0}$ is complete and for any prime $p$, $\ACF_{p}$ is complete.
\end{prop}
\begin{proof}
  The two proofs are similar, so we focus on the characteristic $0$ case.
  According to Vaught's test, we first need to show that $\ACF_{0}$ is consistent,
  which we can do my simply giving a model: the algebraic closure of $\Q$.
  (For $\ACF_{p}$ we take the algebraic closure of $\F_{p}$.)
  We already have all the tools to make such a model:
  \begin{itemize}
    \item Mathlib has definitions of the rationals \texttt{rat} and finite fields \texttt{zmod}.
    \item (I lift them to an arbitrary universe level for generality.)
    \item Mathlib already has a definition of algebraic closure \texttt{algebraic\_closure}.
    \item We showed that any algebraically closed field is a model of $\ACF$
          and that characteristic $n$ fields are models of $\ACF_{n}$.
  \end{itemize}
  \begin{lstlisting}
def algebraic_closure_of_rat :
  Structure ring_signature :=
Rings.struc_to_ring_struc.Structure algebraic_closure.of_ulift_rat

instance algebraic_closure_of_rat_models_ACF : fact (algebraic_closure_of_rat ⊨ ACF) :=
by {split, classical, apply is_alg_closed_to.realize_ACF }

instance : char_zero algebraic_closure_of_rat := ...

theorem algebraic_closure_of_rat_models_ACF₀ :
  algebraic_closure_of_rat ⊨ ACF₀ :=
models_ACF₀_iff.2 ring_char.eq_zero \end{lstlisting}

The next thing to show is that any model of $\ACF_{0}$ is infinite.
This is true since any algebraically closed field is infinite
(I give a proof of this in \texttt{Rings.ToMathlib.algebraic\_closure};
it is just considering the roots of the separable polynomial $x^{n} - 1$ for each $0 < n$):
\begin{lstlisting}
lemma only_infinite_ACF : only_infinite ACF :=
  by { intro M, haveI : fact (M.1 ⊨ ACF) := ⟨ M.2 ⟩, exact is_alg_closed.infinite }\end{lstlisting}

We need categoricity, for a large cardinal.
We choose this to be the continuum $\f{c}$, the cardinality of $\C$.
For each natural there are only finitely many function symbols
of that arity in the language of rings, and of course $\om \le \f{c}$.
It remains to show that any two algebraically closed fields of characteristic $0$
of size $\f{c}$ are isomorphic.
This was proven by Chris Hughes,
and is now available in \texttt{mathlib}.

\begin{lstlisting}
lemma categorical_ACF₀ {κ} (hκ : ω < κ) : fol.categorical κ ACF₀ :=
begin
  intros M N hM hN hMκ hNκ,
  haveI : fact (M ⊨ ACF₀) := ⟨ hM ⟩, haveI : fact (N ⊨ ACF₀) := ⟨ hN ⟩,
  split,
  apply equiv_of_ring_equiv,
  apply classical.choice,
  apply ring_equiv_of_cardinal_eq_of_char_zero, -- credit to Chris Hughes
  repeat { apply_instance },
  repeat { cc },
end\end{lstlisting}

Putting the above together we have
\begin{lstlisting}
theorem is_complete'_ACF₀ : is_complete' ACF₀ :=
is_complete'_of_only_infinite_of_categorical
    instances.algebraic_closure_of_rat
    instances.algebraic_closure_of_rat_models_ACF₀ -- algebraic closure of ℚ is a model of ACF₀
    (only_infinite_subset ACF_subset_ACF₀ only_infinite_ACF) -- alg closed fields are infinite
    -- pick the cardinal κ := 𝔠
    card_functions_omega_le_continuum
    omega_le_continuum
    (categorical_ACF₀ omega_lt_continuum) \end{lstlisting}
\end{proof}

\subsection{Compactness}



\subsection{Proving Lefschetz}
