Returning to model theory of algebraically closed fields.
We begin by introducing the notion of a complete theory:

\begin{dfn}[Complete theories]
    \link{dfn_complete_theory}
    An $L$-theory $T$ is \textit{complete}
    when either of the following equivalent definitions hold:
    \begin{itemize}
      \item  $T$ deduces any sentence of its negation
    \begin{lstlisting}
      def is_complete' (T : Theory L) : Prop :=
      ∀ (ϕ : sentence L), T ⊨ ϕ ∨ T ⊨ ∼ ϕ \end{lstlisting}
      \item Sentences true in any model are deduced by the theory.
    \begin{lstlisting}
      def is_complete'' (T : Theory L) : Prop :=
      ∀ (M : Structure L) (hM : nonempty M) (ϕ : sentence L), M ⊨ T → M ⊨ ϕ → T ⊨ ϕ \end{lstlisting}
      \item All models of $T$ satisfy the same sentences
            (``are elementarily equivalent'').
    \end{itemize}
    Note that the definition \texttt{is\_complete} from the flypitch project
    is stronger than these conditions, and is useful when constructing
    theories with nice properties\footnote{
      Personally, I prefer the word maximal consistent theory for
      their definition \texttt{is\_complete}}.
    However in practice there is no reason to throw that many sentences
    into our language, so we use the versions above.
\end{dfn}
\begin{proof}
  The statement is
\begin{lstlisting}
  lemma is_complete''_iff_is_complete' {T : Theory L} :
    is_complete' T ↔ is_complete'' T := sorry \end{lstlisting}
  The forward direction involves casing on the hypothesis of $T \vDash \phi$
  or $T \vDash \neg \phi$, in the first case we are done,
  and in the second we get a contradiction by
  $\phi$ being both true and false in our model $M$.
\begin{lstlisting}
  { intros H M hM ϕ hMT hMϕ,
    cases H ϕ with hTϕ hTϕ,
    { exact hTϕ },
    {
      have hbot := hTϕ hM hMT,
      rw realize_sentence_not at hbot,
      exfalso,
      exact hbot hMϕ } },
\end{lstlisting}
    On the other hand we need to case on whether $T$
    is consistent or not.
    When $T$ is consistent we can show $T$ deduces
    $\phi$ or its negation by checking in that model,
    otherwise $T$ should deduce anything.
\begin{lstlisting}
  { intros H ϕ,
    by_cases hM : ∃ M : Structure L, nonempty M ∧ M ⊨ T,
    {
      rcases hM with ⟨ M , hM0 , hMT ⟩,
      by_cases hMϕ : M ⊨ ϕ,
      { left, exact H M hM0 ϕ hMT hMϕ },
      {
        right,
        rw ← realize_sentence_not at hMϕ,
        exact H M hM0 _ hMT hMϕ} },
    { left,
      intros M hM0 hMT,
      exfalso,
      apply hM ⟨ M , hM0 , hMT ⟩} } \end{lstlisting}
\end{proof}

\begin{prop}[Lefschetz principle]
    \link{lefschetz}
    Let $\phi$ be a sentence in the language of rings.
    Then the following are equivalent:
    \begin{enumerate}
        \item Some model of $\ACF_0$ satisfies $\phi$.
        (If you like $\C \vDash \phi$.)
        \item $\ACF_0 \vDash \phi$
        \item There exists $n \in \N$ such that for any prime $p$
            greater than $n$, $\ACF_p \vDash \phi$
        \item There exists $n \in \N$ such that for any prime $p$
        greater than $n$, some model of $\ACF_p$ satisfies $\phi$.
    \end{enumerate}
    The first and last equivalences are due to the theories $\ACF_{p}$
    being complete for any $p$ ($0$ or prime).
\end{prop}

To prove the above we need the following
\begin{itemize}
  \item \link{vaught_test}{Vaught's test} for showing a theory is complete
        (this does the first and last equivalences and
        is needed in the middle equivalence)
  \item The compactness theorem for the middle equivalence.
\end{itemize}
In this section we will introduce these notions properly and how
they are used.
Vaught's test will be proven in a \linkto{vaught_proof}{later section}.
The compactness theorem will not be proven
(it was part of the flypitch project).

\subsection{$\ACF_{n}$ is complete (Vaught's test)}
We want to show that $\ACF_{n}$ is complete.
Another way of expressing that a theory $T$ is complete is to
ask for models of $T$ to satisfy the same sentences
(that they are elementarily equivalent).
In particular it is known that isomorphic models %? Ref?
satisfy the same sentences.

\begin{dfn}[Categoricity]
    Given a language $L$ and a cardinal $\ka$,
    an $L$-theory $T$ is called $\ka$-categorical
    if any two models of $T$ of size $\ka$ are isomorphic.

    \begin{lstlisting}
  def categorical (κ : cardinal) (T : Theory L) :=
  ∀ (M N : Structure L) (hM : M ⊨ T) (hN : N ⊨ T), #M = κ → #N = κ → nonempty (M ≃[L] N) \end{lstlisting}
\end{dfn}

Vaught's test says that categoricity is a useful condition for showing a theory is complete.
We check that this holds for $\ACF_{n}$ and uncountable cardinals.
\begin{prop}[Categoricity for $\ACF_{n}$]
  If two algebraically closed fields have the same \textit{uncountable}
  cardinality then they are (non-canonically) isomorphic.
  \begin{lstlisting}
lemma ring_equiv_of_cardinal_eq_of_char_eq
  {K L : Type u} [hKf : field K] [hLf : field L]
  (hKalg : is_alg_closed K) (hLalg : is_alg_closed L)
  (p : ℕ) [char_p K p] [char_p L p]
  (hKω : ω < #K) (hKL : #K = #L) : nonempty (K ≃+* L) := sorry \end{lstlisting}

  Hence $\ACF_{n}$ is $\kappa$-categorical for any uncountable cardinal $\kappa$.
\end{prop}
\begin{proof}
  This is proven by Chris Hughes and is now part of mathlib.
  We outline the argument:

  Let $\F$ be the minimal field in $K$ and $L$,
  which is either finite or $\Q$
  and is the same field since they are of the same characteristic.

  There exist transcendence bases for $K$ and $L$ respectively,
  which we can call $s$ and $t$.
  Since $K$ and $L$ are both uncountable,
  the transcendence bases must be of the same cardinality as the fields.
  \[ \# K = \om + \# s = \# s \quad \text{ and }
    \quad \# L = \om + \# t = \# t \]
  Then $t$ and $s$ biject, hence we have ring isomorphisms
  \[K \iso \F(s) \iso \F(t) \iso L\]

  Then we can apply this to show categoricity:

\begin{lstlisting}
lemma categorical_ACF₀ {κ} (hκ : ω < κ) : fol.categorical κ ACF₀ :=
begin
  intros M N hM hN hMκ hNκ,
  haveI : fact (M ⊨ ACF₀) := ⟨ hM ⟩, haveI : fact (N ⊨ ACF₀) := ⟨ hN ⟩,
  split,
  apply equiv_of_ring_equiv,
  apply classical.choice,
  apply ring_equiv_of_cardinal_eq_of_char_zero, -- the char 0 version of what we showed above
  repeat { apply_instance },
  repeat { cc },
end\end{lstlisting}
\end{proof}

Another condition we will need for Vaught's test
is that there are only infinite models to the theory

\begin{lstlisting}
  def only_infinite (T : Theory L) : Prop := ∀ (M : Model T), infinite M.1\end{lstlisting}

This will hold in our case since algebraically closed fields are infinite.
We are now in a position to state Vaught's test.

\begin{prop}[Vaught's Test]
  \link{vaught_test}
  Let $L$ be a language and $T$ be a consistent theory in the language $L$
  with only infinite models, such that it is $\ka$-categorical
  for some large enough cardinal $\ka$ (see below for details).
  Then $T$ is a complete theory.

  \begin{lstlisting}
  lemma is_complete'_of_only_infinite_of_categorical
    [is_algebraic L] {T : Theory L} (M : Structure L) (hM : M ⊨ T)
    (hinf : only_infinite T) {κ : cardinal}
    (hκ : ∀ n, #(L.functions n) ≤ κ) (hωκ : ω ≤ κ) (hcat : categorical κ T) :
    is_complete' T := sorry
\end{lstlisting}
  This may differ slightly to the statement in other sources;
  the reason for the choice of these (stronger than usual)
  hypotheses will be discussed
  in the \linkto{vaught_proof}{section dedicated it its proof}.
  % In practice it is less work to prove this when $L$ has no relation symbols,
  % which is the case we are interested in (we say $L$ \textit{is algebraic}).
  % Another practical simplification is asking for $\kappa$ to be larger than
  % the collection of all function symbols.
  % The proof that Marker gives %? Ref
  % only requires that $\kappa$ is larger than the collection of constant symbols.
\end{prop}

We apply Vaught's test in our case to show that the theory
of algebraically closed fields of a fixed characteristic is complete.
However, before we do so we need a field theory lemma.


\begin{prop}
  $\ACF_{0}$ is complete and for any prime $p$, $\ACF_{p}$ is complete.
\end{prop}
\begin{proof}
  The two proofs are similar, so we focus on the characteristic $0$ case.
  According to Vaught's test, we first need to show that $\ACF_{0}$ is consistent,
  which we can do my simply giving a model: the algebraic closure of $\Q$.
  (For $\ACF_{p}$ we take the algebraic closure of $\F_{p}$.)
  We already have all the tools to make such a model:
  \begin{itemize}
    \item Mathlib has definitions of the rationals \texttt{rat} and finite fields \texttt{zmod}.
    \item (I lift them to an arbitrary universe level for generality.)
    \item Mathlib already has a definition of algebraic closure \texttt{algebraic\_closure}.
    \item We showed that any algebraically closed field is a model of $\ACF$
          and that characteristic $n$ fields are models of $\ACF_{n}$.
  \end{itemize}
  \begin{lstlisting}
def algebraic_closure_of_rat :
  Structure ring_signature :=
Rings.struc_to_ring_struc.Structure algebraic_closure.of_ulift_rat

instance algebraic_closure_of_rat_models_ACF : fact (algebraic_closure_of_rat ⊨ ACF) :=
by {split, classical, apply is_alg_closed_to.realize_ACF }

instance : char_zero algebraic_closure_of_rat := ...

theorem algebraic_closure_of_rat_models_ACF₀ :
  algebraic_closure_of_rat ⊨ ACF₀ :=
models_ACF₀_iff.2 ring_char.eq_zero \end{lstlisting}

The next thing to show is that any model of $\ACF_{0}$ is infinite.
This is true since any algebraically closed field is infinite
(I give a proof of this in \texttt{Rings.ToMathlib.algebraic\_closure};
it is just considering the roots of the separable polynomial $x^{n} - 1$ for each $0 < n$):
\begin{lstlisting}
lemma only_infinite_ACF : only_infinite ACF :=
  by { intro M, haveI : fact (M.1 ⊨ ACF) := ⟨ M.2 ⟩, exact is_alg_closed.infinite }\end{lstlisting}

We need a large cardinal for categoricity.
We choose this to be the continuum $\f{c}$, the cardinality of $\C$.
It is large enough since for each natural there are only finitely many function symbols
of that arity in the language of rings, and of course $\om \le \f{c}$.

Putting the above together we have
\begin{lstlisting}
theorem is_complete'_ACF₀ : is_complete' ACF₀ :=
is_complete'_of_only_infinite_of_categorical
    instances.algebraic_closure_of_rat
    instances.algebraic_closure_of_rat_models_ACF₀ -- algebraic closure of ℚ is a model of ACF₀
    (only_infinite_subset ACF_subset_ACF₀ only_infinite_ACF) -- alg closed fields are infinite
    -- pick the cardinal κ := 𝔠
    card_functions_omega_le_continuum
    omega_le_continuum
    (categorical_ACF₀ omega_lt_continuum) \end{lstlisting}
\end{proof}

\subsection{Compactness}

One way of stating compactness is the idea that proofs are finite.

\begin{prop}[Compactness (in terms of deduction)]
If $T$ is an $L$-theory and $f$ is an $L$-sentence
then $T$ deduces $\phi$ if and only if there is some finite subtheory of $T$
that deduces $f$.

\begin{lstlisting}
theorem compactness {L : Language} {T : Theory L} {f : sentence L} :
  T ⊨ f ↔ ∃ fs : finset (sentence L), (↑fs : Theory L) ⊨ (f : sentence L) ∧ ↑fs ⊆ T := sorry
\end{lstlisting}
\end{prop}

Confusingly, this can be found in a file called \texttt{completeness.lean}.
The backwards direction of this is obvious since any model of $T$ automatically
is a model of a finite subset.

There is an alternative formulation of compactness which we do not use for Lefschetz,
but is important as a tool for showing that a theory is consistent.
The reader may choose to come back to it when it is referred to later on.

\begin{prop}[Compactness (in terms of consistency)]
  If $T$ is an $L$-theory then $T$ is consistent if and only if
  each finite subtheory of $T$ is consistent.

  \begin{lstlisting}
theorem compactness' {L} {T : Theory L} : is_consistent T ↔
  ∀ fs : finset (sentence L), ↑fs ⊆ T → is_consistent (↑fs : Theory L) := \end{lstlisting}

  Often the term ``finitely consistent'' is used to describe the latter case.
\end{prop}

I prove the second statement (using the lemmas made for the first) in
\texttt{Rings.ToMathlib.completeness.lean}.
The proof I give below is \textit{not exactly} this proof,
since I wish to avoid first order logic syntax ($\vdash$),
which is the default layer of
definitions used in the \texttt{flypitch} library,
and just argue using model theory ($\vDash$).
However the essence of the proof is the same.

\begin{prop}
  The two formulations of compactness are equivalent.
\end{prop}
\begin{proof}
  \begin{forward}
    Clearly if $T$ is consistent with a model $\MM$ then $\MM$
    is also a model of any subtheory of $T$.

    For the converse we prove the contrapositive.
    Suppose $T$ is inconsistent, then $T \vDash \bot$,
    since proving this requires assuming a model of $T$.
    The first version of compactness
    implies there is a finite subset of $T$ that deduces $\bot$.
    This subset cannot be consistent, as any model will satisfy $\bot$.
  \end{forward}

  \begin{backward}
    Clearly if a finite subtheory of $T$ deduces a sentence $\phi$ then
    any model of $T$ is a model of the subtheory, hence also satisfies $\phi$.

    For the converse we again prove the contrapositive.
    \textit{Note that for a theory $\De$ and a sentence $\phi$ we have
      $\De \nvDash \phi$ if and only if $\De \cup \set{\neg \phi}$ is consistent.}
    We make use of this fact:
    Suppose all finite subtheories of $T$ do not deduce a sentence $\phi$.
    Then for any finite subtheory $\De \subs T$, we have $\De \nvDash \phi$ and so
    $\De \cup {\neg \phi}$ is consistent.
    Then $T \cup \set{\neg \phi}$ is a finitely consistent theory,
    hence is consistent by the second version of compactness.
    Hence $T \nvDash \phi$.
  \end{backward}
\end{proof}

\subsection{Proving Lefschetz}

We are now ready to prove the Lefschetz principle.
We begin by showing that if $\ACF_{0}$ deduces a ring sentence $\phi$
then $\ACF_{p}$ deduces $\phi$ for large $p$.
We prove it seperately because it will be used in the converse!

\begin{lstlisting}
theorem characteristic_change_left (ϕ : sentence ring_signature) :
ACF₀ ⊨ ϕ → ∃ (n : ℕ), ∀ {p : ℕ} (hp : nat.prime p), n < p → ACFₚ hp ⊨ ϕ := sorry \end{lstlisting}

\begin{proof}
We apply compactness:
if $\ACF_{0}$ deduces $\phi$ then we must have a finite subtheory of
$\ACF_{0}$ that deduces $\phi$.
In particular since $\ACF_{0}$ consisted of the axioms for $\ACF$ plus
$p + 1 \ne 0$ for each $p \in \N$ we know that only finitely many
such formulas are needed to deduce $\phi$.
Hence our $n$ should be the maximum $p$ such that $p + 1 \ne 0$
is in our finite subtheory, plus $1$.
\begin{lstlisting}
begin
  rw compactness,
  intro hsatis,
  obtain ⟨ fs , hsatis , hsub ⟩ := hsatis,
  classical,
  obtain ⟨ fsACF , fsrange , hunion, hACF , hrange ⟩ :=
    finset.subset_union_elim hsub,
  set fsnat : finset ℕ := finset.preimage fsrange plus_one_ne_zero
      (set.inj_on_of_injective injective_plus_one_ne_zero _) with hfsnat,
  use fsnat.sup id + 1, \end{lstlisting}

In the above \texttt{fs} is our finite subtheory,
\texttt{fsrange} is the part consisting of the formulas $p + 1 \ne 0$,
and the other part from $\ACF$ is called \texttt{fsACF}.

Let us then suppose that we have a prime that is larger than this $n$.
By design,
it should be that $\ACF_{p}$ deduces all the formulas from
our finite subtheory, hence $\ACF_{p}$ should deduce $\phi$.
Supposing that $M$ is a model of $\ACF_{p}$,
it suffices that $M$ deduces $\phi$.
Since $\texttt{fs} = \texttt{fsACF} \cup \texttt{fsrange}$
deduces $\phi$ it suffices that $M$ deduces \texttt{fsrange}
(it deduces $\ACF$ so it deduces any subset of $\ACF$).

\begin{lstlisting}
  intros p hp hlt M hMx hmodel,
  haveI : fact (M ⊨ ACF) := ⟨ (models_ACFₚ_iff'.mp hmodel).2 ⟩,
  have hchar := (@models_ACFₚ_iff _ _ _inst_1 _).1 hmodel,
  apply hsatis hMx,
  rw [← hunion, finset.coe_union, all_realize_sentence_union],
  split,
  { apply all_realize_sentence_of_subset _ hACF,
    exact all_realize_sentence_of_subset hmodel ACF_subset_ACFₚ},
  {sorry}, \end{lstlisting}

It suffices to show that for each $q$ in \texttt{fsnat}
(the list of $q$ such that $q + 1 \ne 0$ is in the subtheory \texttt{fs}),
$M$ satisfies $q + 1 \ne 0$.
We can then conclude that this is true since $M$ is characteristic $p$,
and $q < p$.
\end{proof}

Now for the whole theorem:

\begin{lstlisting}
theorem characteristic_change (ϕ : sentence ring_signature) :
ACF₀ ⊨ ϕ ↔ (∃ (n : ℕ), ∀ {p : ℕ} (hp : nat.prime p), n < p → ACFₚ hp ⊨ ϕ) := sorry \end{lstlisting}
\begin{proof}
It remains to prove the converse.
We know that $\ACF_{0}$ is complete,
so either $\ACF_{0} \vDash \phi$ or $\ACF_{0} \vDash \neg \phi$,
and it suffices to refute the latter case.

\begin{lstlisting}
begin
  split,
  { apply characteristic_change_left },
  {
    intro hn,
    cases is_complete'_ACF₀ ϕ with hsatis hsatis,
    { exact hsatis },
    { sorry }, \end{lstlisting}

We can apply the forward direction of Lefschetz,
and have that $\ACF_{p}$ deduces $\neg \phi$ for large $p$.
We instantiate these lower bounds, and take any prime $p$ that is larger
than their maximum.

\begin{lstlisting}
    { have hm := characteristic_change_left (∼ ϕ) hsatis,
      cases hn with n hn,
      cases hm with m hm,
      obtain ⟨ p , hle , hp ⟩ := nat.exists_infinite_primes (max n m).succ, \end{lstlisting}

We take the algebraic closure $\Om$ of $\F_{p}$ as a model of $\ACF_{p}$,
and since $p$ is suitably large, we have that $\Om \vDash \phi$ and
$\Om \vDash \neg \phi$, which is a contradiction.

\begin{lstlisting}
      have hnp : n < p :=
        lt_of_lt_of_le (nat.lt_succ_of_le (le_max_left _ _)) hle,
      have hmp : m < p :=
        lt_of_lt_of_le (nat.lt_succ_of_le (le_max_right _ _)) hle,
      have hS := instances.algebraic_closure_of_zmod_models_ACFₚ hp, -- Ω ⊨ ACFₚ
      specialize @hn p hp hnp _ ⟨ 0 ⟩ hS,
      specialize @hm p hp hmp _ ⟨ 0 ⟩ hS,
      simp only [realize_sentence_not] at hm,
      exfalso,
      apply hm hn } },
end \end{lstlisting}

\end{proof}
