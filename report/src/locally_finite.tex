Since Chris Hughes wrote the proof to this part of the project
I will only explain the mathematics behind the proof and not
talk about the \texttt{lean} formalization of it.

\begin{dfn}[Locally finite fields \cite{stack0}]
    \link{locally_finite}
    Let $K$ be a field of characteristic $p$ a prime.
    Then the following are equivalent definitions for $K$ being a
    \textit{locally finite field}:
    \begin{enumerate}
        \item The minimal subfield generated by any finite subset of $K$ is finite.
        \item $\F_p \to K$ is algebraic.
        \item $K$ embeds into an algebraic closure of $\F_p$.
    \end{enumerate}
    The important example for us of a locally finite field is an algebraic closure of $\F_{p}$.
    By the following theorem, this is a model of $\ACF_{p}$ satisfying Ax-Grothendieck.
\end{dfn}
\begin{proof}
  $1.\implies 2.$
  Let $a \in K$. Then $\F_{p}(a)$ is the minimal subfield generated by $a$,
  and is finite by assumption.
  Finite field extensions are algebraic $a$ is algebraic over $\F_{p}$.

  $2. \implies 1.$ We show by induction that $K$ is locally finite.
  Let $S$ be a finite subset of $K$.
  If $S$ is empty then $\F_p(S) = \F_p$ and so it is finite.
  If $S = T \cup {s}$ and $\F_p(T)$ is finite,
  then $s \in K$ is algebraic so by some basic field theory we can
  take the quotient by the minimal polynomial of $s$ giving
  \[\F_p(T)[x] / \min (s, \F_p(T)) \iso \F_p(S)\]
  The left hand side is finite because it is a finite
  dimensional vector space over a finite field.
  Hence $K$ is locally finite.

  $2. \iff 3.$ These are basic properties of algebraic closures.
\end{proof}

\begin{prop}[Ax-Grothendieck for locally finite fields]
  \link{ax_groth_locally_fin}
  Let $L$ be a locally finite field. Then any injective
  \linkto{dfn_poly_map}{polynomial map} $f : L^{n} \to L^{n}$ is surjective.
\end{prop}
\begin{proof}
  Let $b = (b_{1},\dots,b_{n}) \in L^{n}$.
  Writing $f = (f_1, \dots, f_n)$ for $f_i \in L[x_1, \dots, x_n]$
  we can find $A \subs L$,
  the set of all the coefficients of all of the $f_i$ when written out in monomials.
  $A \cup \set{b_{1},\dots,b_{n}}$ is finite and $L$ is locally finite,
  so the subfield $K$ generated by it is also finite.

  The restriction $\res{f}{K^n}$ is injective and has image inside $K^n$
  since each polynomial has coefficients in $K$ and is evaluated at
  an element of $K^n$.
  An injective endomorphism of a finite set is surjective,
  hence $b \in K^{n} = f(K^{n})$.
\end{proof}

It is important to note that surjectivity does not imply
injectivity for locally finite fields.
An example:
\[ P = x^{2} + x + 1 : \bar{\F}_{2} \to \bar{\F}_{2}\]
is a polynomial map between locally finite fields.
This is surjective since the algebraic closure has all
roots of all polynomials over it.
It is \textit{not injective} since this polynomial is
separable in characteristic two (its derivative is a unit).

The reason any imitation of the above proof will fail
in this direction is that the restriction of a surjective polynomial map
is not in general surjective.
Let $a \in \bar{F}_{2}$ be a root of $x^{2} + x + 1$. Then
\[ P |_{\F(a)} : \F(a) \to \F(a) \]
takes $0,1 \mapsto 1$ and $a, a + 1 \mapsto 0$.
Note that $\F(a)$ is size $4$ so we have shown it is
neither injective nor surjective.
