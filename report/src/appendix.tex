\subsection{Relational languages}

\begin{dfn}[Language of binary relations]
  \link{dfn_bin_rel}
  Let the following be the language of binary relations:
    \begin{itemize}
        \item There are no function symbols.
        \item The only relation symbol is $<$, which has arity $2$.
    \end{itemize}
    For variables $x$ and $y$ of type $P$,
    we write $x < y$ as notation for $<(x,y)$.
\end{dfn}

\begin{dfn}[Graph theory]
  \link{graph_theory}
  The theory of simple directed graphs is the language of binary relations is
  given by the single formula
  \begin{itemize}
    \item Non-reflexivity - $\forall x, \NOT (x < x)$
  \end{itemize}
  Here the sort symbol is viewed as the set of vertices,
  and the relation symbol viewed as an edge between vertices.
  The fact that any model of this theory is a simple graph
  follows from non-reflexivity and proposition extensionality.

  The theory of simple undirected graphs includes the formula above and also
  \begin{itemize}
    \item Symmetry - $\forall x \, y, x < y \to y < x$
  \end{itemize}
\end{dfn}

\begin{dfn}[Order theories]
    \link{order_theories}
    The theory of partial orders in the language of binary relations
    is given by the two formulas
    \begin{itemize}
        \item Non-reflexivity - $\forall x, \NOT (x < x)$
        \item Transitivity - $\forall x \, y \, z, x < y \AND y < z \implies x < z$
    \end{itemize}
    The theory of linear orders is the theory of partial orders plus
    \begin{itemize}
        \item Linearity - $\forall x \, y, x < y \OR x = y \OR y < x$
    \end{itemize}
\end{dfn}

\begin{dfn}[Theory of equivalence relations]
    \link{equiv_rel_theory}
    The theory of equivalence relations in the language of binary relations
    is given by the three formulas
    \begin{itemize}
      \item Reflexivity - $\forall x, x < x$
      \item Symmetry - $\forall x \, y \, z, x < y \implies y < x$
      \item Transitivity - $\forall x \, y \, z, x < y \AND y < z \implies x < z$
    \end{itemize}
\end{dfn}

ZFC can also be described as a theory in the language of binary relations.
The above theories all contain finitely many sentences,
but one would need infinitely many sentences to make ZFC,
since each axiom schema includes infinitely many sentences.

\subsection{Actions}

\begin{dfn}[Monoid and group actions]
  \link{monoid_action_dfn}
  Fix $M$ a monoid.
  Let the following be the language of $M$-sets:
  \begin{itemize}
    \item For each monoid element $a$ from $M$
          we have a function symbol $\rho_{a}$ of arity $1$.
    \item There are no relation symbols.
  \end{itemize}

  In the language of $M$-sets we let the theory of left monoid actions consist of
  the sentences
  \begin{itemize}
    \item Respecting multiplication - $\set{ \forall x, \rho_{a}(\rho_{b} x) = \rho_{ab} \st a , b \in M}$
    \item Respecting the identity - $ \forall x, \rho_{e} x = x$, where $e$ is the identity in $M$
  \end{itemize}

  The above automatically defines $G$-sets and left $G$ actions for a group $G$.
\end{dfn}

\begin{dfn}[Modules]
  \link{dfn_module}
  Fix $A$ a ring.
  Let the language of modules over $A$ be the \linkto{language_sum_dfn}{sum}
  of the \linkto{monoid_action_dfn}{language of $A$-sets} and the language of groups.
  (We already made the language of groups and the theory of abelian groups
  whilst making the language of rings and theory of commutative rings.)

  In the langauge of modules we make the theory of left modules by taking the
  union of the following
  \begin{itemize}
    \item The (\linkto{induced_theory_dfn}{induced}) theory of left monoid actions
          from the language of $A$-sets.
    \item The (\linkto{induced_theory_dfn}{induced}) theory of abelian groups
          from the language of groups.
    \item Ring and module distributivity:
          \[
            \set{ \forall x \, y, \rho_{a} (x + y) = \rho_{a} x + \rho_{a} y \st a \in A }
            \cup
            \set{ \forall x, \rho_{a + b} x = \rho_{a} x + \rho_{b} x \st a , b \in A }
          \]
  \end{itemize}
\end{dfn}

\subsection{Inductive types}

\begin{dfn}[The naturals]
  \link{nat_dfn}

  The naturals convey the data of their induction principle,
  which is embodied in the following.

  \begin{lstlisting}
  inductive nat
  | zero : nat
  | succ (n : nat) : nat \end{lstlisting}
  This definition of the naturals reads
  ``the only things of type \code{nat} are
  \code{nat.zero} and \code{nat.succ n},
  for some \code{n} of type \code{nat}''.
\end{dfn}

The recursor for the naturals says to make a map out of the naturals
into some type \code{A} (i.e. recursively make things of type \code{A}),
it suffices to send \code{nat.zero} to something of type \code{A}
and, given the image of \code{n},
to send \code{nat.succ n} to something of type \code{A}.

\begin{dfn}[Lists]
  \link{list_dfn}

  Given a type \code{T}, we can build lists of things of type \code{A}.

  \begin{lstlisting}
  inductive list (T : Type u)
  | nil : list
  | cons (hd : T) (tl : list) : list \end{lstlisting}

  This definition of lists on a type \code{T} reads
  ``the only things of type \code{list T} are
  \code{list.nil} and \code{list.cons hd tl},
  for some \code{hd} of type \code{T}
  and some \code{tl} of type \code{list T}''.
  The first constructor is the empty list and the
  second gives a way of making a new list by adding \code{hd}
  at the beginning of an existing list.
\end{dfn}

The recursor for lists says to make a map out of lists on a type
into some type \code{A} (i.e. recurse on lists),
it suffices to send the empty list to something of type \code{A}
and, given the image of a list \code{tl},
to send \code{list.cons hd tl} to something of type \code{A}.
