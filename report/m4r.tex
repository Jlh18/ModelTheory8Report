\documentclass{article}
\usepackage[tmargin = 30mm,bmargin = 30mm]{geometry}
% \usepackage[left=1in,right=1in]{geometry}
\usepackage{subfiles}
\usepackage{amsmath, amssymb, stmaryrd, verbatim, bbm} % math symbols
\usepackage{amsthm} % thm environment
\usepackage{mdframed} % Customizable Boxes
\usepackage{hyperref,nameref,cleveref,enumitem} % for references, hyperlinks
\usepackage[dvipsnames]{xcolor} % Fancy Colours
\usepackage{mathrsfs} % Fancy font
\usepackage{tikz, tikz-cd, float} % Commutative Diagrams
\usepackage{perpage}
\usepackage{parskip} % So that paragraphs look nice
\usepackage{ifthen,xargs} % For defining better commands
\usepackage{anyfontsize}
\usepackage[T1]{fontenc}
\usepackage[utf8]{inputenc}
\usepackage{tgpagella}
\usepackage{titlesec}
\usepackage{url}
\usepackage{listings}

% % Set the monospace font
\usepackage{inconsolata}

% % Misc
\newcommand{\brkt}[1]{\left(#1\right)}
\newcommand{\sqbrkt}[1]{\left[#1\right]}
\newcommand{\dash}{\text{-}}
\newcommand{\tdt}{\times \dots \times}

% % Logic
\renewcommand{\implies}{\Rightarrow}
\renewcommand{\iff}{\Leftrightarrow}
\newcommand{\IFF}{\leftrightarrow}
\newcommand{\limplies}{\Leftarrow}
\newcommand{\NOT}{\neg\,}
\newcommand{\AND}{\land}
\newcommand{\OR}{\lor}
\newenvironment{forward}{($\implies$)}{}
\newenvironment{backward}{($\limplies$)}{}
% General way of making larger symbols with limits above and below
\makeatletter
\DeclareRobustCommand\bigop[1]{%
  \mathop{\vphantom{\sum}\mathpalette\bigop@{#1}}\slimits@
}
\newcommand{\bigop@}[2]{%
  \vcenter{%
    \sbox\z@{$#1\sum$}%
    \hbox{\resizebox{
      \ifx#1\displaystyle.7\fi\dimexpr\ht\z@+\dp\z@}{!}{$\m@th#2$}}% symbol size
  }%
}
\makeatother
\newcommand{\bigforall}[2]{\DOTSB\bigop{\forall}_{#1}^{#2}}
\newcommand{\bigexists}[2]{\DOTSB\bigop{\exists}_{#1}^{#2}}
\newcommand{\bigand}[2]{\DOTSB\bigop{\mbox{\Large$\land$}}_{#1}^{#2}}
\newcommand{\bigor}[2]{\DOTSB\bigop{\mbox{\Large$\lor$}}_{#1}^{#2}}

% % Sets
\DeclareMathOperator{\supp}{supp}
\newcommand{\set}[1]{\left\{#1\right\}}
\newcommand{\st}{\,|\,}
\newcommand{\minus}{\setminus}
\newcommand{\subs}{\subseteq}
\newcommand{\ssubs}{\subsetneq}
\DeclareMathOperator{\im}{Im}
\newcommand{\nothing}{\varnothing}
\newcommand\res[2]{{% we make the whole thing an ordinary symbol
  \left.\kern-\nulldelimiterspace
  % automatically resize the bar with \right
  #1 % the function
  \vphantom{\big|}
  % pretend it's a little taller at normal size
  \right|_{#2} % this is the delimiter
  }}

% % Greek
\newcommand{\al}{\alpha}
\newcommand{\be}{\beta}
\newcommand{\ga}{\gamma}
\newcommand{\de}{\delta}
\newcommand{\ep}{\varepsilon}
\newcommand{\io}{\iota}
\newcommand{\ka}{\kappa}
\newcommand{\la}{\lambda}
\newcommand{\om}{\omega}
\newcommand{\si}{\sigma}

\newcommand{\Ga}{\Gamma}
\newcommand{\De}{\Delta}
\newcommand{\Th}{\Theta}
\newcommand{\La}{\Lambda}
\newcommand{\Si}{\Sigma}
\newcommand{\Om}{\Omega}

% % Mathbb
\newcommand{\A}{\mathbb{A}}
\newcommand{\N}{\mathbb{N}}
\newcommand{\M}{\mathbb{M}}
\newcommand{\Z}{\mathbb{Z}}
\newcommand{\Q}{\mathbb{Q}}
\newcommand{\R}{\mathbb{R}}
\newcommand{\C}{\mathbb{C}}
\newcommand{\F}{\mathbb{F}}
\newcommand{\V}{\mathbb{V}}
\newcommand{\U}{\mathbb{U}}

% % Mathcal
\renewcommand{\AA}{\mathcal{A}}
\newcommand{\BB}{\mathcal{B}}
\newcommand{\CC}{\mathcal{C}}
\newcommand{\DD}{\mathcal{D}}
\newcommand{\EE}{\mathcal{E}}
\newcommand{\FF}{\mathcal{F}}
\newcommand{\GG}{\mathcal{G}}
\newcommand{\HH}{\mathcal{H}}
\newcommand{\II}{\mathcal{I}}
\newcommand{\JJ}{\mathcal{J}}
\newcommand{\KK}{\mathcal{K}}
\newcommand{\LL}{\mathcal{L}}
\newcommand{\MM}{\mathcal{M}}
\newcommand{\NN}{\mathcal{N}}
\newcommand{\OO}{\mathcal{O}}
\newcommand{\PP}{\mathcal{P}}
\newcommand{\QQ}{\mathcal{Q}}
\newcommand{\RR}{\mathcal{R}}
\renewcommand{\SS}{\mathcal{S}}
\newcommand{\TT}{\mathcal{T}}
\newcommand{\UU}{\mathcal{U}}
\newcommand{\VV}{\mathcal{V}}
\newcommand{\WW}{\mathcal{W}}
\newcommand{\XX}{\mathcal{X}}
\newcommand{\YY}{\mathcal{Y}}
\newcommand{\ZZ}{\mathcal{Z}}

% % Mathfrak
\newcommand{\f}[1]{\mathfrak{#1}}

% % Mathrsfs
\newcommand{\s}[1]{\mathscr{#1}}

% % Category Theory
\newcommand{\obj}[1]{\mathrm{Obj}\left(#1\right)}
\newcommand{\Hom}[3]{\mathrm{Hom}_{#3}(#1, #2)\,}
\newcommand{\mor}[3]{\mathrm{Mor}_{#3}(#1, #2)\,}
\newcommand{\End}[2]{\mathrm{End}_{#2}#1\,}
\newcommand{\aut}[2]{\mathrm{Aut}_{#2}#1\,}
\newcommand{\CAT}{\mathbf{Cat}}
\newcommand{\SET}{\mathbf{Set}}
\newcommand{\TOP}{\mathbf{Top}}
%\newcommand{\GRP}{\mathbf{Grp}}
\newcommand{\RING}{\mathbf{Ring}}
\newcommand{\MOD}[1][R]{#1\text{-}\mathbf{Mod}}
\newcommand{\VEC}[1][K]{#1\text{-}\mathbf{Vec}}
\newcommand{\ALG}[1][R]{#1\text{-}\mathbf{Alg}}
\newcommand{\PSH}[1]{\mathbf{PSh}\brkt{#1}}
\newcommand{\map}[2]{ \yrightarrow[#2][2.5pt]{#1}[-1pt] }
\newcommand{\op}{^{op}}
\newcommand{\darrow}{\downarrow}
\newcommand{\LIM}[2]{\varprojlim_{#2}#1}
\newcommand{\COLIM}[2]{\varinjlim_{#2}#1}
\newcommand{\hookr}{\hookrightarrow}

% % Algebra
\newcommand{\iso}{\cong}
\newcommand{\nsub}{\trianglelefteq}
\newcommand{\id}[1]{\mathbbm{1}_{#1}}
\newcommand{\inv}{^{-1}}
\DeclareMathOperator{\dom}{dom}
\DeclareMathOperator{\codom}{codom}
\DeclareMathOperator{\coker}{Coker}
\DeclareMathOperator{\spec}{Spec}

% % Analysis
\newcommand{\abs}[1]{\left\vert #1 \right\vert}
\newcommand{\norm}[1]{\left\Vert #1 \right\Vert}
\renewcommand{\bar}[1]{\overline{#1}}
\newcommand{\<}{\langle}
\renewcommand{\>}{\rangle}
\renewcommand{\check}[1]{\widecheck{#1}}

% % Galois
\newcommand{\Gal}[2]{\mathrm{Gal}_{#1}(#2)}
\DeclareMathOperator{\Orb}{Orb}
\DeclareMathOperator{\Stab}{Stab}
\newcommand{\emb}[3]{\mathrm{Emb}_{#1}(#2, #3)}
\newcommand{\Char}[1]{\mathrm{Char}#1}

% % Model Theory
\newcommand{\intp}[2]{
    \star_{\text{\scalebox{0.7}{$#1$}}}^{
    \text{\scalebox{0.7}{$#2$}}}}
\newcommand{\subintp}[3]{
    {#3}_{\text{\scalebox{0.7}{$#1$}}}^{
    \text{\scalebox{0.7}{$#2$}}}}
\newcommand{\modintp}[2]{#2^\text{\scalebox{0.7}{$#1$}}}
\newcommand{\mmintp}[1]{\modintp{\MM}{#1}}
\newcommand{\nnintp}[1]{\modintp{\NN}{#1}}
\DeclareMathOperator{\const}{constants}
\DeclareMathOperator{\func}{functions}
\DeclareMathOperator{\rel}{relations}
\newcommand{\term}[1]{{#1}_\mathrm{ter}}
% \newcommand{\tv}[1]{\textrm{tv}_{#1}}
% \newcommand{\struc}[1]{\mathbf{Str}(#1)}
% \newcommand{\form}[1]{{#1}_\mathrm{for}}
% \newcommand{\var}[1]{{#1}_\mathrm{var}}
% \newcommand{\theory}[1]{{#1}_\mathrm{the}}
% \newcommand{\carrier}[1]{{#1}_\mathrm{car}}
% \newcommand{\model}[1]{\vDash_{#1}}
% \newcommand{\nodel}[1]{\nvDash_{#1}}
% \newcommand{\modelsi}{\model{\Si}}
% \newcommand{\nodelsi}{\nvDash_{\Si}}
% \newcommand{\eldiag}[2]{\mathrm{ElDiag}(#1,#2)}
% \newcommand{\atdiag}[2]{\mathrm{AtDiag}(#1,#2)}
% \newcommand{\Theory}{\mathrm{Th}}
% \newcommand{\unisen}[1]{{#1}_\mathrm{uni}}
\newcommand{\lift}[2]{\uparrow_{#1}^{#2}}
\newcommand{\fall}[2]{\downarrow_{#1}^{#2}}
\DeclareMathOperator{\GRP}{GRP}
\newcommand{\RNG}{\mathrm{RNG}}
\newcommand{\ER}{\mathrm{ER}}
\DeclareMathOperator{\FLD}{FLD}
\DeclareMathOperator{\ID}{ID}
\DeclareMathOperator{\ZFC}{ZFC}
\DeclareMathOperator{\ACF}{ACF}
\newcommand{\BLN}{\mathrm{BLN}}
\newcommand{\PO}{\mathrm{PO}}
\DeclareMathOperator{\tp}{tp}
\DeclareMathOperator{\qftp}{qftp}
\DeclareMathOperator{\qf}{qf}
\DeclareMathOperator{\eqzero}{eqzero}
\newcommand{\MR}[2]{\mathrm{MR}^{#1}(#2)}
\DeclareMathOperator{\MD}{MD}
\DeclareMathOperator{\acl}{acl}
\DeclareMathOperator{\cl}{cl}
\DeclareMathOperator{\mdeg}{m.deg}
\DeclareMathOperator{\kdim}{k.dim}
\newcommand{\Mod}[1]{EDITTHIS {#1}}
% % Set theory
\DeclareMathOperator{\ord}{Ord}

% % Boolean algebra
\newcommand{\NEG}{\smallsetminus}
\newcommand{\upa}[1]{#1^{\uparrow}}

% % Field theory
\DeclareMathOperator{\tdeg}{t.deg}
\newcommand{\zmo}[2][p]{\Z/#1^{#2}\Z}

%% code from mathabx.sty and mathabx.dcl to get some symbols from mathabx
\DeclareFontFamily{U}{mathx}{\hyphenchar\font45}
\DeclareFontShape{U}{mathx}{m}{n}{
      <5> <6> <7> <8> <9> <10>
      <10.95> <12> <14.4> <17.28> <20.74> <24.88>
      mathx10
      }{}
\DeclareSymbolFont{mathx}{U}{mathx}{m}{n}
\DeclareFontSubstitution{U}{mathx}{m}{n}
\DeclareMathAccent{\widecheck}{0}{mathx}{"71}

% Arrows with text above and below with adjustable displacement
% (Stolen from Stackexchange)
\newcommandx{\yaHelper}[2][1=\empty]{
\ifthenelse{\equal{#1}{\empty}}
  % no offset
  { \ensuremath{ \scriptstyle{ #2 } } }
  % with offset
  { \raisebox{ #1 }[0pt][0pt]{ \ensuremath{ \scriptstyle{ #2 } } } }
}

\newcommandx{\yrightarrow}[4][1=\empty, 2=\empty, 4=\empty, usedefault=@]{
  \ifthenelse{\equal{#2}{\empty}}
  % there's no text below
  { \xrightarrow{ \protect{ \yaHelper[ #4 ]{ #3 } } } }
  % there's text below
  {
    \xrightarrow[ \protect{ \yaHelper[ #2 ]{ #1 } } ]
    { \protect{ \yaHelper[ #4 ]{ #3 } } }
  }
}

% xcolor
\definecolor{darkgrey}{gray}{0.10}
\definecolor{lightgrey}{gray}{0.30}
\definecolor{slightgrey}{gray}{0.80}
\definecolor{softblue}{RGB}{30,100,200}

% hyperref
\hypersetup{
      colorlinks = true,
      linkcolor = {softblue},
      citecolor = {blue}
}

\newcommand{\link}[1]{\hypertarget{#1}{}}
\newcommand{\linkto}[2]{\hyperlink{#1}{#2}}

% Theorems

% % custom theoremstyles
\newtheoremstyle{definitionstyle}
{0pt}% above thm
{0pt}% below thm
{}% body font
{}% space to indent
{\bf}% head font
{\vspace{1mm}}% punctuation between head and body
{\newline}% space after head
{\thmname{#1}\thmnote{\,\,--\,\,#3}}

% % custom theoremstyles
\newtheoremstyle{propositionstyle}
{0pt}% above thm
{0pt}% below thm
{}% body font
{}% space to indent
{\bf}% head font
{\vspace{1mm}}% punctuation between head and body
{\newline}% space after head
{\thmname{#1}\thmnote{\,\,--\,\,#3}}

\newtheoremstyle{exercisestyle}%
{0pt}% above thm
{0pt}% below thm
{\it}% body font
{}% space to indent
{\scshape}% head font
{.}% punctuation between head and body
{ }% space after head
{\thmname{#1}\thmnote{ (#3)}}

\newtheoremstyle{remarkstyle}%
{0pt}% above thm
{0pt}% below thm
{}% body font
{}% space to indent
{\it}% head font
{.}% punctuation between head and body
{ }% space after head
{\thmname{#1}\thmnote{\,\,--\,\,#3}}

% % Theorem environments

\theoremstyle{definitionstyle}
\newmdtheoremenv[
    %skipabove = \baselineskip
    linewidth = 2pt,
    leftmargin = 3pt,
    rightmargin = 0pt,
    linecolor = darkgrey,
    topline = false,
    bottomline = false,
    rightline = false,
    footnoteinside = true
]{dfn}{Definition}
\newmdtheoremenv[
    linewidth = 2 pt,
    leftmargin = 3pt,
    rightmargin = 0pt,
    linecolor = darkgrey,
    topline = false,
    bottomline = false,
    rightline = false,
    footnoteinside = true
]{prop}{Proposition}
\newmdtheoremenv[
    linewidth = 2 pt,
    leftmargin = 3pt,
    rightmargin = 0pt,
    linecolor = darkgrey,
    topline = false,
    bottomline = false,
    rightline = false,
    footnoteinside = true
]{cor}{Corollary}
\newmdtheoremenv[
    linewidth = 2 pt,
    leftmargin = 3pt,
    rightmargin = 0pt,
    linecolor = darkgrey,
    topline = false,
    bottomline = false,
    rightline = false,
    footnoteinside = true
]{lem}{Lemma}


\theoremstyle{exercisestyle}
\newmdtheoremenv[
    linewidth = 0.7 pt,
    leftmargin = 20pt,
    rightmargin = 0pt,
    linecolor = darkgrey,
    topline = false,
    bottomline = false,
    rightline = false,
    footnoteinside = true
]{ex}{Exercise}
\newmdtheoremenv[
    linewidth = 0.7 pt,
    leftmargin = 3pt,
    rightmargin = 0pt,
    linecolor = darkgrey,
    topline = false,
    bottomline = false,
    rightline = false,
    footnoteinside = true
]{eg}{Example}
\newmdtheoremenv[
    linewidth = 0.7 pt,
    leftmargin = 3pt,
    rightmargin = 0pt,
    linecolor = darkgrey,
    topline = false,
    bottomline = false,
    rightline = false,
    footnoteinside = true
]{nttn}{Notation}

\theoremstyle{remarkstyle}
\newtheorem{rmk}{Remark}

% % footnotes
\renewcommand{\thempfootnote}{$\dagger$}
\MakePerPage{footnote}

% % tikzcd diagram
\newenvironment{cd}{
    \begin{figure}[H]
    \centering
    \begin{tikzcd}
}{
    \end{tikzcd}
    \end{figure}
}

% tikzcd
% % Substituting symbols for arrows in tikz comm-diagrams.
\tikzset{
  symbol/.style={
    draw=none,
    every to/.append style={
      edge node={node [sloped, allow upside down, auto=false]{$#1$}}}
  }
}

\titlespacing*{\section}
{0pt}{5ex plus 1ex minus .2ex}{4ex plus .2ex}
\titlespacing*{\subsection}
{0pt}{5ex plus 1ex minus .2ex}{4ex plus .2ex}

% Syntax highlighting for lean

\usepackage{color}
\definecolor{keywordcolor}{rgb}{0.6, 0.3, 0.6}   % light purple
\definecolor{tacticcolor}{rgb}{0.2, 0.2, 0.6}    % dark purple
\definecolor{commentcolor}{rgb}{0.4, 0.4, 0.4}   % grey
\definecolor{symbolcolor}{rgb}{0.7, 0.2, 0.2}    % blue
\definecolor{sortcolor}{rgb}{0.2, 0.7, 0.7}      % cyan
\definecolor{attributecolor}{RGB}{120,0,0}       % maroon
\definecolor{draculaGrey}{RGB}{25, 27, 15}       % grey

\def\lstlanguagefiles{lstlean.tex}
% set default language
\lstset{language=lean}

\begin{document}
\title{Ax-Grothendieck and Lean}
\author{Joseph Hua}
\maketitle

\tableofcontents

\newpage
\section{Introduction}
It is a basic fact of linear algebra that any linear map
between vector spaces of the same finite dimension is
injective if and only if it is surjective.
\linkto{ax_groth_thm}{Ax-Grothendieck} says that this is
partly true for \linkto{dfn_poly_map}{polynomial maps}.

Here are some examples of polynomial maps
\begin{itemize}
  \item Surjective but not injective: $f : \C \to \C := x \mapsto x^{2}$
  \item Neither surjective nor injective:
      $f : \C^{2} \to \C^{2} := (x,y) \mapsto (x,xy)$
  \item Bijective:
      $f : \C^{3} \to \C^{3} := (x,y,z) \mapsto ( x , y , z + xy )$
\end{itemize}

One might try very hard to look for an example of an injective polynomial map
that is not surjective.
Replacing $\C$ with a finite field, we notice that
surjectivity and injectivity coincide in this case.
Ax-Grothendieck states that over any algebraically closed field,
injectivity of a polynomial map implies surjectivity,
and the proof we will give is model theoretic,
roughly saying ``we may reduce to the finite field case''.

This project formalizes this proof of Ax-Grothendieck in \texttt{lean}\footnote{
  As far as I am aware this is the first attempt to formalize Ax-Grothendieck in
  proof verification software. }.
My github repository for this project is accessible at
\url{github.com/Jlh18/ModelTheoryInLean8}.

I work on a fork of the \texttt{flypitch} project \cite{flypitch},
which built the framework for basic model theory and proved the
independence of the continuum hypothesis.
I have made many updates to this fork to make it compatible with
more modern versions of \texttt{mathlib}.
My original work all lies in the folder \texttt{src/Rings},
even though a lot of the code (such as the proof of \linkto{vaught_test}{Vaught's test})
is general model theory.


\section{Model Theory Background}
For most definitions and proofs in this section we reference
David Marker's book on Model Theory \cite{marker}.
We introduce the formalisations of the content in \texttt{lean} alongside the theory,
walking through the basics of definitions made in the \texttt{flypitch} project \cite{flypitch}.
My work is based on a slightly updated (3.33.0) version of the flypitch project,
combined with some of the model theory material put in \texttt{mathlib} (which was edited for compatibility).

\subsection{Languages}
\begin{dfn}[Language]
  A language (also known as a \textit{signature}) $\LL = ( \func , \rel )$ consists of

  \begin{itemize}
    \item A sort symbol $A$, which we will have in the background for intuition.
    \item For each natural number $n$ we have $\func n$ -
          the set of \textit{function symbols} of \textit{arity} $n$ for the language.
          For some $f \in \func n$ we might write
          $f : A^{n} \to A$ to denote $f$ with its arity.
    \item For each natural number $n$ we have $\rel n$ -
          the set of \textit{relation symbols} of \textit{arity} $n$ for the language.
          For some $r \in \rel n$ we might write
          $r \hookr A^{n}$ to denote $r$ with its arity.
  \end{itemize}

  The \texttt{flypitch} project implements the above definition as

  \begin{lstlisting}
    structure Language : Type (u+1) :=
      (functions : ℕ → Type u)
      (relations : ℕ → Type u)\end{lstlisting}

  This says that \texttt{Language} is a mathematical structure
  (like a group structure, or ring structure)
  that consists of two pieces of data,
  a map called \texttt{functions} and another called \texttt{relations}.
  Both take a natural number and spit out a \textit{type}
  (which in \texttt{lean} might as well mean \textit{set})
  that consists respectively of all the function symbols and relation symbols of arity $n$.

  In more detail: in type theory when we write \texttt{a:A} we mean \texttt{a} is something
  of \textit{type} \texttt{A}.
  We can draw an analogy with the set theoretic notion $a \in A$,
  but types in \texttt{lean} have slightly different
  personalities, which we will gradually introduce.
  Hence in the above definitions \texttt{functions n} and \texttt{relations n}
  are things of type \texttt{Type u}.
  \texttt{Type u} is a collection of all types at level \texttt{u},
  so things of type \texttt{Type u} are types.
  ``Types are type \texttt{Type u}.''

  For convenience we single out $0$-ary (arity $0$) functions and
  call them \textit{constant} symbols, usually denoting them by $c : A$.
  We think of these as `elements' of the sort $A$ and write $c : A$.
  This is defined in \texttt{lean} by

  \begin{lstlisting}
    def constants (L : Language) : Type u := functions 0\end{lstlisting}

  This says that \texttt{constants} takes in a language $L$ and returns a type.
  Following the \texttt{:=} we have the definition of \texttt{constants L},
  which is the type \texttt{functions 0}.
\end{dfn}

\begin{eg}
  The \linkto{dfn_rings}{language of rings}
  will be used to define the theory of rings,
  the theory of integral domains, the theory of fields, and so on.
  In the appendix we give examples:
  \begin{itemize}
    \item The \linkto{dfn_bin_rel}{language with just a single binary relation} %? missing
          can be used to define the theory of partial orders
          with the interpretation of the relation as $<$,
          to define the theory of equivalence relations with the
          interpretation of the relation as $\sim$,
          and to define the theory $\ZFC$ with the relation interpreted as $\in$.
    \item The \linkto{dfn_cat}{language of categories} %? missing
          can be used to define the theory of categories.
    \item The \linkto{dfn_graph}{language of simple graphs}
          can be used to define the theory of simple graphs
  \end{itemize}
  We will only be concerned with the language of rings and will
  focus our examples around this.
\end{eg}

\begin{dfn}[Language of rings]
    \link{dfn_rings}
    Let the following be the language of rings:
    \begin{itemize}
        \item The function symbols are the constant symbols $0, 1 : A$,
        the symbols for addition and multiplication $+ , \times : A^2 \to A$
        and taking for inverse $- : A \to A$.
        \item There are no relation symbols.
    \end{itemize}

    We can break this definition up into steps in \texttt{lean}.
    We first collect the constant, unary and binary symbols:

    \begin{lstlisting}
      /-- The constant symbols in RingLanguage -/
      inductive ring_consts : Type u
      | zero : ring_consts
      | one : ring_consts

      /-- The unary function symbols in RingLanguage-/
      inductive ring_unaries : Type u
      | neg : ring_unaries

      /-- The binary function symbols in RingLanguage-/
      inductive ring_binaries : Type u
      | add : ring_binaries
      | mul : ring_binaries\end{lstlisting}

    These are \textit{inductively defined types} -
    types that are `freely' generated by their constructors,
    listed below after each bar `\texttt{|}'.
    In these above cases they are particularly simple -
    the only constructors are terms in the type.
    In the appendix we give more examples of inductive types %? missing
    \begin{itemize}
      \item The \link{dfn_nat}{natural numbers} are defined as inductive types
      \item \link{dfn_list}{Lists} are defined as inductive types
      \item The \link{dfn_int}{integers} can be defined as inductive types
    \end{itemize}

    We now collect all the above into a single definition \texttt{ring funcs}
    that takes each natural \texttt{n} to the type of \texttt{n}-ary
    function symbols in the language of rings.

    \begin{lstlisting}
      /-- All function symbols in RingLanguage-/
      def ring_funcs : ℕ → Type u
      | 0 := ring_consts
      | 1 := ring_unaries
      | 2 := ring_binaries
      | (n + 3) := pempty\end{lstlisting}

    The type \texttt{pempty} is the empty type and is meant to have no terms in it,
    since we wish to have no function symbols beyond arity $2$.
    Finally we make the language of rings

    \begin{lstlisting}
      /-- The language of rings -/
      def ring_language : Language :=
      (Language.mk) (ring_funcs) (λ n, pempty)\end{lstlisting}

\end{dfn}

We use languages to express logical assertions about our structures, such as
``any degree two polynomial over my ring has a root''.
In order to do so we must introduce terms (polynomials in our case),
formulas (the assertion itself), structures and models (the ring),
and the relation between structures and formulas
(that the ring satisfies this assertion).

We want to express ``all the combinations of symbols we can make in a language''.
We can think of multivariable polynomials over the integers as such:
the only things we can write down using symbols $0,1,-,+,*$ and variables
are elements of $\Z{[x_{k}]}_{k \in \N}$.
We formalize this as terms.

\subsection{Terms and formulas}

\begin{dfn}[Terms]
  Let $\LL = (\func, \rel)$ be a language.
  To make a \textit{preterm} in $\LL$ with up to $n$ variables
  we can do one of three things:
  \begin{itemize}
    \item[$\vert$] For each natural number $k < n$ we create a symbol
          $x_{k}$, which we call a \textit{variable} in $A$.
          Any $x_{k}$ is a preterm (that is missing nothing).
    \item[$\vert$] If $f : A^{l} \to A$ is a function symbol then
          $f$ is a preterm that is missing $l$ inputs.
          \[ f( ? , \cdots , ? )\]
    \item[$\vert$] If $t$ is a preterm that is missing
          $l + 1$ inputs and $s$ is a preterm that is missing
          no inputs then we can \textit{apply} $t$ to $s$, obtaining
          a preterm that is missing $l$ inputs.
          \[ t(s , ? , \cdots, ? )\]
  \end{itemize}

  We only really want \textit{terms} with up to $n$ variables,
  which are defined as preterms that are missing nothing.

  \begin{lstlisting}
    inductive bounded_preterm (n : ℕ) : ℕ → Type u
    | x_ : ∀ (k : fin n), bounded_preterm 0
    | bd_func : ∀ {l : ℕ} (f : L.functions l), bounded_preterm l
    | bd_app : ∀ {l : ℕ} (t : bounded_preterm (l + 1)) (s : bounded_preterm 0), bounded_preterm l

  def bounded_term (n : ℕ) := bounded_preterm L n 0\end{lstlisting}

  To explain notation
  \begin{itemize}
    \item The second constructor says ``for all natural numbers $l$ and function symbols $f$,
          \texttt{bd\_func f} is something in \texttt{bounded\_preterm l}''.
          This makes sense since \texttt{bounded\_preterm l} is a type by the first line of code.
    \item The curley brackets just say
          ``you can leave out this input and \texttt{lean} will know what it is''.
  \end{itemize}

  To give an example of this in action we can write $x_{1} * 0$.
  We first write the individual parts, which are
  \texttt{x\_ 1}, \texttt{bd\_func mul} and
  \texttt{bd\_func zero}.
  Then we apply them to each other
  \begin{lstlisting}
    bd_app (bd_app (mul) (x_ 1)) zero \end{lstlisting}
  Naturally, we will introduce nice notation in \texttt{lean} to replace all of this.
\end{dfn}

\begin{rmk}
  There are many terminology clashes between model theory and type theory,
  since they are closely related.
  The word ``term'' in type theory refers to anything on the left of a \texttt{:} sign,
  or anything in a type.
  Terms in inductively defined types are (as mentioned before)
  freely generated symbols using the contructors.
  Analogously terms in a language are freely generated symbols using
  the symbols from the language.
\end{rmk}

One can imagine writing down any degree two polynomial over the integers
as a term in the language of rings.
In fact, we could even make degree two polynomials over any ring (if we had one):
\[ x_{0} x_{3}^{2} + x_{1} x_{3} + x_{2} \]
Here our variable is $x_{3}$, and we imaging that the other variables represent
elements of our ring.

To express ``any degree (up to) two polynomial over our ring has a root'',
we might write
\[ \forall x_{2} x_{1} x_{0} : A, \exists x_{3} : A, x_{0} x_{3}^{2} + x_{1} x_{3} + x_{2} = 0 \]
Formulas allow us to do this.

\begin{dfn}[Formulas]
  Let $\LL$ be a language.
  A (classical first order) $\LL$-\textit{preformula} in $\LL$
  with (up to) $n$ \textit{free} variables can be built in the following ways:
  \begin{itemize}
    \item[$\vert$] $\bot$ is an atomic preformula with $n$ free variables
          (and missing nothing).
    \item[$\vert$]
          Given terms $t, s$ with $n$ variables,
          $t = s$ is a formula with $n$ free variables (missing nothing).
    \item[$\vert$] Any relation symbol $r \hookr A^{l}$ is a preformula
          with $n$ free variables and missing $l$ inputs.
          \[ r (?, \cdots, ?)\]
    \item[$\vert$] If $\phi$ is a preformula with $n$ free variables that is missing
          $l + 1$ inputs and $t$ is a term with $n$ variables
          then we can \textit{apply} $\phi$ to $t$, obtaining
          a preformula that is missing $l$ inputs.
          \[ \phi(t , ? , \cdots, ? )\]
    \item[$\vert$] If $\phi$ and $\psi$ are preformulas with $n$ free variables
          and \textit{nothing missing} then so is $\phi \implies \psi$.
    \item[$\vert$] If $\phi$ is a preformula with $n + 1$ free variables
          and \textit{nothing missing} then $\forall x_{0}, \phi$ is a preformula
          with $n$ free variables and nothing missing.
  \end{itemize}

  We take formulas to be preformulas with nothing missing.
  Note that we take the de Brujn index convension here.
  If $\phi$ were the formula $x_{0} + x_{1} = x_{2}$ then $\forall \phi$ would be
  the formula $\forall x_{0} : A, x_{0} + x_{1} = x_{2}$,
  which is really $\forall x : A, x + x_{0} = x_{1}$,
  so that all the remaining free variables are shifted down.

  We write this in \texttt{lean},
  and also define sentences as preformulas with $0$ variables and nothing missing.
  Sentences are what we usually come up with when we make assertions.
  For example $x = 0$ is not an assersion about rings,
  but $\forall x : A, x = 0$ is.

  \begin{lstlisting}
    inductive bounded_preformula : ℕ → ℕ → Type u
    | bd_falsum {n : ℕ} : bounded_preformula n 0
    | bd_equal {n : ℕ} (t₁ t₂ : bounded_term L n) : bounded_preformula n 0
    | bd_rel {n l : ℕ} (R : L.relations l) : bounded_preformula n l
    | bd_apprel {n l : ℕ} (f : bounded_preformula n (l + 1)) (t : bounded_term L n) : bounded_preformula n l
    | bd_imp {n : ℕ} (f₁ f₂ : bounded_preformula n 0) : bounded_preformula n 0
    | bd_all {n : ℕ} (f : bounded_preformula (n+1) 0) : bounded_preformula n 0

    def bounded_formula (n : ℕ) := bounded_preformula L n 0
    def sentence := bounded_preformula L 0 0\end{lstlisting}

  Since we are working with classical logic we
  make everything else we need by use of the excluded middle:

  \begin{lstlisting}
    /-- ⊥ is for bd_falsum, ≃ for bd_equal, ⟹ for bd_imp, and ∀' for bd_all -/
    /-- we will write ~ for bd_not, ⊓ for bd_and, and infixr ⊔ for bd_or -/
    def bd_not {n} (f : bounded_formula L n) : bounded_formula L n := f ⟹ ⊥
    def bd_and {n} (f₁ f₂ : bounded_formula L n) : bounded_formula L n := ~(f₁ ⟹ ∼f₂)
    def bd_or {n} (f₁ f₂ : bounded_formula L n) : bounded_formula L n := ~f₁ ⟹ f₂
    def bd_biimp {n} (f₁ f₂ : bounded_formula L n) : bounded_formula L n := (f₁ ⟹ f₂) ⊓ (f₂ ⟹ f₁)
    def bd_ex {n} (f : bounded_formula L (n+1)) : bounded_formula L n := ~ (∀' ~ f))
  \end{lstlisting}
\end{dfn}

With this set up we can already write down the sentences that describe rings.
\link{sentences_for_ring_theory}

\begin{lstlisting}
  /-- Assosiativity of addition -/
  def add_assoc : sentence ring_signature :=
  ∀' ∀' ∀' ( (x_ 0 + x_ 1) + x_ 2 ≃ x_ 0 + (x_ 1 + x_ 2) )

  /-- Identity for addition -/
  def add_id : sentence ring_signature := ∀' ( x_ 0 + 0 ≃ x_ 0 )

  /-- Inverse for addition -/
  def add_inv : sentence ring_signature := ∀' ( - x_ 0 + x_ 0 ≃ 0 )

  /-- Commutativity of addition-/
  def add_comm : sentence ring_signature := ∀' ∀' ( x_ 0 + x_ 1 ≃ x_ 1 + x_ 0 )

  /-- Associativity of multiplication -/
  def mul_assoc : sentence ring_signature :=
  ∀' ∀' ∀' ( (x_ 0 * x_ 1) * x_ 2 ≃ x_ 0 * (x_ 1 * x_ 2) )

  /-- Identity of multiplication -/
  def mul_id : sentence ring_signature :=  ∀' ( x_ 0 * 1 ≃ x_ 0 )

  /-- Commutativity of multiplication -/
  def mul_comm : sentence ring_signature := ∀' ∀' ( x_ 0 * x_ 1 ≃ x_ 1 * x_ 0   )

  /-- Distributibity -/
  def add_mul : sentence ring_signature :=
  ∀' ∀' ∀' ( (x_ 0 + x_ 1) * x_ 2 ≃ x_ 0 * x_ 2 + x_ 1 * x_ 2 )\end{lstlisting}

We later collect all of these into one set and call it the \linkto{ring_theory}{theory of rings}.

\subsection{Interpretation of symbols}

In the above we set up a symbol treatment of logic.
In this subsection we try to make these symbols into tangible mathematical objects.

We intend to apply the statement
``any degree two polynomial over our ring has a root''
to a real, usable, tangible ring.
We would like the sort symbol $A$ to be interpreted as the underlying type (set)
for the ring and the function symbols to actually become maps from the ring to itself.

\begin{dfn}[Structures]
    Given a language $\LL$, a $\LL$-\textit{structure} \texttt{M}
    interpreting $\LL$ consists of the following
    \begin{itemize}
      \item An underlying type \texttt{carrier}.
      \item Each function symbol $f : A^{n} \to A$ is interpreted as a
            function that takes an $n$-ary tuple in \texttt{carrier}
            to something in \texttt{carrier}.
      \item Each relation symbol $r \hookr A^{n}$
            is interpreted as a proposition about $n$-ary tuples in \texttt{carrier},
            which can also be viewed as the subset of the set of $n$-ary tuples
            satisfying that proposition.
    \end{itemize}

  \begin{lstlisting}
  structure Structure :=
  (carrier : Type u)
  (fun_map : ∀{n}, L.functions n → dvector carrier n → carrier)
  (rel_map : ∀{n}, L.relations n → dvector carrier n → Prop)\end{lstlisting}

  The \texttt{flypitch} library uses \texttt{dvector A n} for $n$-ary tuples of terms in \texttt{A}.

  Note that rather comically \texttt{Structure} is itself a mathematical structure.
  This is sensible, since \texttt{Structure} is meant to generalize the algebraic
  (and relational) definitions of mathematical structures such as groups and rings.

  Also note that for constant symbols the interpretation has domain empty tuples,
  i.e. only the term \texttt{dvector.nil} as its domain. Hence it is a constant map
  - a term of the interpreted carrier type.
\end{dfn}

The structures in a language will become the models of \linkto{dfn_theory}{theories}.
For example $\Z$ is a structure in the \linkto{dfn_rings}{language of rings},
a model of the \linkto{ring_theory}{theory of rings} but not a model of the theory of fields.
In the language of \linkto{dfn_bin_rel}{binary relations},
$\N$ with the usual ordering $\leq$ is a structure that models of
the theory of partial orders (with the order relation)
but not the theory of equivalence relations (with $\le$).

Before continuing on formalizing ``any degree two polynomial over our ring has a root'',
we stop to make the remark that the collection of all structures in a language forms a category.
To this end we define morphisms of structures.

\begin{dfn}[$\LL$-morphism, $\LL$-embedding]
    \link{category_of_structures}
    The collection of all $\LL$-structures forms a category with objects
    as $\LL$-structures and morphisms as $\LL$-morphisms.

    \begin{lstlisting}
protected structure hom :=
(to_fun : M → N)
(map_fun' : ∀{n} (f : L.functions n) x, to_fun (M.fun_map f x)
  = N.fun_map f (dvector.map to_fun x) . obviously)
(map_rel' : ∀{n} (r : L.relations n) x, M.rel_map r x
  → N.rel_map r (dvector.map to_fun x) . obviously)\end{lstlisting}

    The induced map between the $n$-ary tuples is called \texttt{dvector.map}.
    The above says a morphism is a mathematical structure
    consisting of three pieces of data.
    The first says that we have a functions between the carrier types,
    the second gives a sensible commutative diagram for functions,
    and the last gives a sensible commutative diagram for relations\footnote{
      The way to view relations on a structure categorically is to view it
      as a subobject of the carrier type.}.

    \begin{cd}
      \texttt{dvector M.carrier n}
      \ar[r, "\texttt{M.fun\_map}"] \ar[d, "\texttt{dvector.map to\_fun}", swap]
      & \texttt{M.carrier} \ar[d, "\texttt{to\_fun}"]\\
      \texttt{dvector N.carrier n}
      \ar[r, "\texttt{N.fun\_map}"] & \texttt{N.carrier}\\
        \mmintp{r}
        \ar[hookrightarrow]{r} \ar[d, "\texttt{dvector.map to\_fun}", swap]
        & \texttt{dvector M.carrier n}
        \ar[d, "\texttt{dvector.map to\_fun}"]\\
        \nnintp{r}
        \ar[hookrightarrow]{r} & \texttt{dvector N.carrier n}
      \end{cd}

      The notion of morphisms here will be the same as that of
      morphisms in the algebraic setting.
      For example in the \linkto{dfn_rings}{language of rings},
      preserving interpretation of function symbols says
      the zero is sent to the zero, one is sent to one,
      subtraction, multiplication and addition is preserved.
      In languages that have relation symbols,
      such as that of simple graphs, preserving relations says that
      if the relation holds for terms in the domain,
      then the relation holds for their images.
\end{dfn}

Returning to our objective,
we realize that we need to interpret our degree two polynomial (a term)
is something in our ring. The term

\[ x_{0} x_{3}^{2} + x_{1} x_{3} + x_{2} \]
Should be a map from $4$-tuples from the ring to a value in the ring,
namely, taking $(a, b, c, d)$ to

\[ a c^{2} + b c + d \]

We thus need to figure out how terms in the language interact with
structures in the language.

\begin{dfn}[Interpretation of terms]
    \link{interpretation_terms}
    Given $\LL$-structure $\MM$ and a $\LL$-term $t$ with up to $n$-variables.
    Then we can naturally interpret (a.k.a realize) $t$ in the $\LL$-structure $\MM$ as a
    map from the $n$-tuples of $\MM$ to $\MM$ that
    commutes with the interpretation of function symbols.

    \begin{lstlisting}
@[simp] def realize_bounded_term {M : Structure L} {n} (v : dvector M n) :
  ∀{l} (t : bounded_preterm L n l) (xs : dvector M l), M.carrier
| _ (x_ k)         xs := v.nth k.1 k.2
| _ (bd_func f)    xs := M.fun_map f xs
| _ (bd_app t₁ t₂) xs := realize_bounded_term t₁ (realize_bounded_term t₂ ([])::xs) \end{lstlisting}

    This is defined by induction on (pre)terms.
    When the preterm $t$ is a variable $x_{k}$, we interpret $t$ as a map
    that picks out the $k$-th part of the $n$-tuple \texttt{xs}.
    This is like projecting to the $n$-th axis if the structure looks like an affine line.
    When the term is a function symbol, then we automatically get a map from the
    definition of structures.
    In the last case we are applying a preterm $t_{1}$ to a term $t_{2}$,
    and by induction we already have interpretation of these two preterms
    in our structure, so we compose these in the obvious way.
\end{dfn}

We can finally completely formalize
``any (at most) degree two polynomial has a root''.

\begin{dfn}[Interpretation of formulas]

    Given $\LL$-structure $\MM$ and a $\LL$-formula $f$ with up to $n$-variables.
    Then we can interpret (a.k.a realize or satisfy) $f$ in the $\LL$-structure $\MM$ as a
    proposition about $n$ terms from the carrier type.

    \begin{lstlisting}
@[simp] def realize_bounded_formula {M : Structure L} :
  ∀{n l} (v : dvector M n) (f : bounded_preformula L n l) (xs : dvector M l), Prop
| _ _ v bd_falsum       xs := false
| _ _ v (t₁ ≃ t₂)       xs := realize_bounded_term v t₁ xs = realize_bounded_term v t₂ xs
| _ _ v (bd_rel R)      xs := M.rel_map R xs
| _ _ v (bd_apprel f t) xs := realize_bounded_formula v f (realize_bounded_term v t ([])::xs)
| _ _ v (f₁ ⟹ f₂)       xs := realize_bounded_formula v f₁ xs → realize_bounded_formula v f₂ xs
| _ _ v (∀' f)          xs := ∀(x : M), realize_bounded_formula (x::v) f xs \end{lstlisting}

  This is defined by induction on (pre)formulas.
  \begin{itemize}
    \item[$\vert$] $\bot$ is interpreted as the type theoretic proposition \texttt{false}.
    \item[$\vert$] $t = s$ is interpreted as type theoretic equality of the interpreted terms.
    \item[$\vert$] Interpretation of relation symbols is part of the data of an
          $\LL$-structure (\texttt{rel\_map}).
    \item[$\vert$] If $f$ is a preformula with $n$ free variables that is missing
          $l + 1$ inputs and $t$ is a term with $n$ variables
          then $f$ applied to $t$ can be interpreted using the interpretation of $f$ and
          applied to the interpretation of $t$, both of which are given by induction.
    \item[$\vert$] An implication can be interpreted as a type theoretic implication
          using the inductively given interpretations on each formula.
    \item[$\vert$] $\forall x_{0}, f$ can be interpreted as the type theoretic proposition
          ``for each $x$ in the carrier set $P$'',
          where $P$ is the inductively given interpretation.
  \end{itemize}

  We write $\MM \models f(a)$ to mean ``the realization of $f$ holds in $\MM$ for the terms $a$''.
  We are particularly interested in the case when the formula is a sentence,
  which we denote as $\MM \models f$ (since we need no terms).
  \begin{lstlisting}
@[reducible] def realize_sentence (M : Structure L) (f : sentence L) : Prop :=
realize_bounded_formula ([] : dvector M 0) f ([])\end{lstlisting}
\end{dfn}

Now we are able to express ``this structure in the language of rings has roots
of all degree two polynomials'', using interpretation of sentences.

\subsection{Theories}

%?missing a description

\begin{dfn}[Theory]
  \link{dfn_theory}
  Given a language $\LL$,
  a set of sentences in the language is a theory in that language.
  \begin{lstlisting}
    def Theory := set (sentence L)   \end{lstlisting}
\end{dfn}


\begin{dfn}[Models]
    Given an $\LL$-structure $\MM$ and $\LL$-theory $T$,
    we write $\MM \models T$ and say
    \emph{$\MM$ is a model of $T$} when
    for all sentences $f \in T$ we have $\MM \models f$.

    \begin{lstlisting}
      def all_realize_sentence (M : Structure L) (T : Theory L) := ∀ f, f ∈ T → M ⊨ f \end{lstlisting}
  \end{dfn}

A model of the \linkto{ring_theory}{theory of rings} should be exactly the data of a ring.
Before \linkto{algebraic_objects_iff_models}{converting between algebraic objects
  and their model theoretic counterparts}, so we first write down the theories of
rings, fields, and algebraically closed fields.

\begin{dfn}[The theories of rings, fields and algebraically closed fields]
  \link{ring_theory}
  The theory of rings is just the set of the
  \linkto{sentences_for_ring_theory}{sentences describing a ring}
\begin{lstlisting}
def ring_theory : Theory ring_signature :=
{add_assoc, add_id, add_inv, add_comm, mul_assoc, mul_id, mul_comm, add_mul}\end{lstlisting}

  To make the theory of fields we can add two sentences saying that
  the ring is non-trivial and has multiplicative inverses:
  \begin{lstlisting}
def mul_inv : sentence ring_signature :=
∀' (x_ 1 ≃ 0) ⊔ (∃' x_ 1 * x_ 0 ≃ 1)

def non_triv : sentence ring_signature := ~ (0 ≃ 1)

def field_theory : Theory ring_signature := ring_theory ∪ {mul_inv , non_triv} \end{lstlisting}

  To make the theory of algebraically closed fields we need to express
  ``every non-constant polynomial has a root''.
  We replace this with the equivalent statement ``every monic polynomial has a root''.
  We do this by first making ``generic polynomials''
  in the form of $a_{n+1}x^{n} + \cdots + a_{2}x + a_{1}$,
  then adding $x^{n+1}$ to it, making it a ``generic monic polynomial''.
  The (polynomial) variable $x$ will be represented by the variable \texttt{x\_ 0},
  and the coefficient $a_{k}$ for each $0 < k$ will be represented by the variable
  \texttt{x\_ k}.

  We define generic polynomials of degree (at most) $n$ as bounded ring signature terms
  in $n + 2$ variables by induction on $n$:
  when the degree is $0$, we just take the constant polynomial $x_{1}$
  and supply a proof that $1 < 0 + 2$ (we omit these below using underscores).
  When the degree is $n + 1$, we can take the previous generic polynomial,
  lift it up from a term in $n + 2$ variables to $n + 3$ variables
  (this is \texttt{lift\_succ}),
  then add $x_{n + 2} x_{0}^{n+1}$ at the front.

  \begin{lstlisting}
def gen_poly : Π (n : ℕ), bounded_ring_term (n + 2)
| 0       := x_ ⟨ 1 , _ ⟩
| (n + 1) := (x_ ⟨ n + 2 , _ ⟩) * (npow_rec (n + 1) (x_ ⟨ 0 , _ ⟩))
  + bounded_preterm.lift_succ (gen_poly n)
\end{lstlisting}

  Since the type of terms in the language of rings has notions of
  addition and multiplication (using the function symbols),
  we automatically have a way of taking (natural number) powers.
  This is \texttt{npow\_rec}.

  We proceed to making generic monic polynomials by adding
  $x_{0}^{n+2}$ at the front of the generic polynomial.

  \begin{lstlisting}
def gen_monic_poly (n : ℕ) : bounded_term ring_signature (n + 2) :=
npow_rec (n + 1) (x_ 0) + gen_poly n

/-- ∀ a₁ ⋯ ∀ aₙ, ∃ x₀, (aₙ x₀ⁿ⁻¹ + ⋯ + a₂ x₀+ a₁ = 0) -/
def all_gen_monic_poly_has_root (n : ℕ) : sentence ring_signature :=
fol.bd_alls (n + 1) (∃' gen_monic_poly n ≃ 0) \end{lstlisting}

  We can then easily state ``all generic monic polynomials have a root''.
  The order of the variables is important here:
  the $\exists$ removes the first variable $x_{0}$ in the $n+2$ variable formula
  $\texttt{gen\_monic\_poly} n \simeq 0$, and moves the index of all the
  variables down by $1$, making the remaining expression
  $\exists \texttt{gen\_monic\_poly} n \simeq 0$ a formula in $n+1$ variables.
  The function \texttt{fol.bd\_alls n} then adds $n+1$ many ``foralls''
  in front, leaving us a formula with no free variables, i.e. sentence.

  \begin{lstlisting}
/-- The theory of algebraically closed fields -/
def ACF : Theory ring_signature := field_theory ∪ (set.range all_gen_monic_poly_has_root)\end{lstlisting}

  Since \texttt{all\_gen\_monic\_poly\_has\_root} is a function from the naturals,
  we can take its set theoretic image (called \texttt{set.range}),
  i.e. a sentence for each degree $n$ saying
  ``any monic polynomial of degree $n$ has a root''.
\end{dfn}

\begin{prop}[Algebraic objects $\IFF$ models]
  \link{algebraic_objects_iff_models}
  The following are true
  \begin{itemize}
    \item A type $A$ is a ring (according to lean) if and only if
          $A$ is a structure in the language of rings
          that models the theory of rings.
    \item A type $A$ is a field (according to lean) if and only if
          it is a model of the theory of fields.
    \item A type $A$ is an algebraically closed field (of characteristic p)
          if and only if it is a model of $\ACF_{(p)}$.
  \end{itemize}
  For the purposes of design in lean it is
  more sensible to split each ``if and only if'' into seperate constructions,
  for converting the algebraic objects into their model theoretic counterparts
  and vice versa.
  Although these are very obvious facts on paper,
  converting between them takes a bit of work in \texttt{lean},
  especially for the last,
  where some ground work needs to be done for interpreting \texttt{gen\_monic\_poly}.
\end{prop}

% \begin{dfn}[Consequence]
%     Given a $\LL$-theory $T$
%     and a $\LL$-sentence $\phi$,
%     we say $\phi$ is a consequence of $T$
%     and say $T \model{\LL} \phi$
%     when for all $\LL$-models $\MM$ of $T$,
%     we have $\MM \model{\LL} \phi$.
%     We also write $T \model{\LL} \De$
%     for $\LL$-theories $T$ and $\De$
%     when for every $\phi \in \De$ we have $T \model{\LL} \phi$.
% \end{dfn}

% \begin{ex}[Logical consequence]
%     Let $T$ be a $\LL$-theory and $\phi$ and $\psi$ be $\LL$-sentences.
%     Show that the following are equivalent:
%     \begin{itemize}
%         \item $T \model{\LL} \phi \to \psi$
%         \item $T \model{\LL} \phi$ implies $T \model{\LL} \psi$.
%     \end{itemize}
% \end{ex}

% \begin{dfn}[Consistent theory]
%     \link{consistent}
%     A $\LL$-theory $T$ is consistent if either of the following equivalent
%     definitions hold:
%     \begin{itemize}
%         \item
%             There does not exists a
%             $\LL$-sentence $\phi$ such that
%             $T \model{\LL} \phi$ and $T \model{\LL} \NOT \phi$.
%         \item There exists
%             a $\LL$-model of $T$.
%     \end{itemize}
%     Thus the definition of consistent is intuitively
%     `$T$ does not lead to a contradiction'.
%     A theory $T$ is finitely consistent if all
%     finite subsets of $T$ are consistent.
%     This will turn out to be another equivalent definition,
%     given by the \linkto{compactness}{compactness theorem}.
% \end{dfn}
% \begin{proof}
%     We show that the two definitions are equivalent.
%     \begin{forward}
%         Suppose no model exists.
%         Take $\phi$ to be the $\LL$-sentence $\top$.
%         Hence all $\LL$-models of $T$ satisfy $\top$ and $\bot$
%         (there are none) so
%         $T \model{\LL} \top$ and $T \model{\LL} \bot$.
%     \end{forward}
%     \begin{backward}
%         Suppose $T$ has a $\LL$-model $\MM$
%         and $T \model{\LL} \phi$ and $T \model{\LL} \NOT \phi$.
%         This implies $\MM \model{\LL} \phi$ and $\MM \nodel{\LL} \phi$,
%         a contradiction.
%     \end{backward}
% \end{proof}

% \begin{dfn}[Elementary equivalence]
%     Let $\MM$, $\NN$ be $\LL$-structures.
%     They are elementarily equivalent if for any $\LL$-sentence $\phi$,
%     $\MM \model{\LL} \phi$ if and only if $\NN \model{\LL} \phi$.
%     We write $\MM \equiv_\LL \NN$.
% \end{dfn}

% \begin{dfn}[Maximal and complete theories]
%     \link{equiv_def_completeness_0}
%     A $\LL$-theory $T$ is \textit{maximal} if
%     for any $\LL$-sentence $\phi$,
%     $\phi \in T$ or $\NOT \phi \in T$.

%     $T$ is \textit{complete}
%     when either of the following equivalent
%     definitions hold:
%     \begin{itemize}
%         \item For any $\LL$-sentence $\phi$,
%             $T \model{\LL} \phi$ or
%             $T \model{\LL} \NOT \phi$.
%         \item All models of $T$ are elementarily equivalent.
%     \end{itemize}
%     Note that maximal theories are complete.
% \end{dfn}
% \begin{proof}
%     \begin{forward}
%         Let $\MM$ and $\NN$ be models of $T$
%         and $\phi$ be a $\LL$-sentence.
%         If $T \model{\LL} \phi$ then both satisfy $\phi$.
%         Otherwise $\NOT \phi \in T$ and neither satisfy $\phi$.
%     \end{forward}

%     \begin{backward}
%         If $\phi$ is a $\LL$-sentence then suppose for a contradiction
%         \[T \nodel{\LL} \phi \text{ and } T \nodel{\LL} \NOT \phi\]
%         Then there exist models of $T$
%         such that $\MM \nodel{\LL} \phi$ and $\NN \nodel{\LL} \NOT \phi$.
%         By assumption they are elementarily equivalent and so
%         $\MM \model{\LL} \NOT \phi$ implies $\NN \model{\LL} \NOT \phi$,
%         a contradiction.
%     \end{backward}
% \end{proof}

% \begin{ex}[Not consistent, not complete]
%     \link{not_consequence}
%     Let $T$ be a $\LL$-theory
%     and $\phi$ is a $\LL$-sentence.
%     Show that $T \nodel{\LL} \phi$
%     if and only if $T \cup \set{ \NOT \phi}$ is consistent.
%     Furthermore, $T \nodel{\LL} \NOT \phi$
%     if and only if $T \cup \set{\phi}$ is consistent.

%     Note that by definition for $\LL$-structures and
%     $\LL$-formulas we (classically) have that
%     \[
%         \MM \modelsi \NOT \phi(a) \iff \MM \nodelsi \phi(a)
%     \]
%     Find examples of theories that do not satisfy
%     \[
%         T \modelsi \NOT \phi \iff T \nodelsi \phi
%     \]
% \end{ex}


\section{Internal completeness and soundness for ring theories}
Here we list some general facts and tips about working with models:
\begin{itemize}
  \item Things are easier to prove in models, so our proofs tend to
        first translate everything we can to the ring,
        then prove the property there,
        making use of existing lemmas in the library for rings.
  \item An important instance of the above phenomenon is the lack
        of algebraic structure for \texttt{bounded\_ring\_terms}.
        For example, addition for polynomials written as terms
        is \textit{not commutative} until it is interpreted into a structure
        satisfying commutativity,
        even though it is true in a polynomial ring.
  \item Sometimes there is extra definitional rewriting that needs to happen,
        and \texttt{dsimp} (or something similar) is needed alongside \texttt{simp}.
\end{itemize}

\subsection{Ring Structures}

We first make the very obvious observation that given
the \texttt{lean} instances of \texttt{[has\_zero]} and \texttt{[has\_one]}
in some type \texttt{A},
we can make interpretations of the symbols \texttt{ring\_consts.zero}
and \texttt{ring\_consts.one}.
Similarly for the other symbols:

\begin{lstlisting}
def const_map [has_zero A] [has_one A] : ring_consts → dvector A 0 → A
| ring_consts.zero _ := 0
| ring_consts.one  _ := 1

def unaries_map [has_neg A] : ring_unaries → (dvector A 1) → A
| ring_unaries.neg a := - (dvector.last a)

-- Induction on both ring_binaries and dvector
def binaries_map [has_add A] [has_mul A] : ring_binaries → (dvector A 2) → A
| ring_binaries.add (a :: b) := a + dvector.last b
| ring_binaries.mul (a :: b) := a * dvector.last b

def func_map [has_zero A] [has_one A] [has_neg A] [has_add A] [has_mul A] :
  Π (n : ℕ), (ring_funcs n) → (dvector A n) → A
| 0       := const_map
| 1       := unaries_map
| 2       := binaries_map
| (n + 3) := pempty.elim\end{lstlisting}

This allows us to make any type with such instances a ring structure:

\begin{lstlisting}
def Structure : Structure ring_signature :=
Structure.mk A func_map (λ n, pempty.elim)\end{lstlisting}

Conversely given any ring structure,
we can easily pick out the above instances.
For example
\begin{lstlisting}
def add {M : Structure ring_signature} (a b : M.carrier) : M.carrier := @Structure.fun_map _ M 2 ring_binaries.add ([a , b])

instance : has_add M := ⟨ add ⟩\end{lstlisting}

\subsection{Rings}
If $A$ is a ring, then surely it is a model of the theory of rings.
I have supplied \texttt{simp} with enough lemmas to reduce the definitions
until requiring the corresponding property about rings,
and I have chosen the sentences to replicate the format of
each property from \texttt{mathlib}.
For example \texttt{add\_comm} below
is the internal property for the type \texttt{A}
(it is not visible to \texttt{simp}),
and it looks exactly like the statement $\texttt{M} \vDash \texttt{add\_comm}$.

\begin{lstlisting}
lemma realize_ring_theory :
  (Structure A) ⊨ ring_signature.ring_theory :=
  /- Structure A : is A as a ring structure as defined above -/
begin
  intros ϕ h,
  repeat {cases h}, /- we are checking A ⊨ ϕ for each ϕ -/
  { intros a b c,
    simp [add_assoc] },
  { intro a,
    simp },
  { intro a,
    simp },
  { intros a b,
    simp [add_comm] },
  { intros a b c,
    simp [mul_assoc] },
  { intro a,
    simp [mul_one] },
  { intros a b,
    simp [mul_comm] },
  { intros a b c,
    simp [add_mul] }
end\end{lstlisting}

Conversely, given a model of the theory of rings
we can supply an instance of a ring to the carrier type.
I supply a lemma for each piece of data going into a \texttt{comm\_ring}.
As an example, we look at \texttt{add\_comm}.

\begin{lstlisting}
/- First show that add_comm is in ring_theory -/
lemma add_comm_in_ring_theory : add_comm ∈ ring_theory :=
begin apply_rules [set.mem_insert, set.mem_insert_of_mem] end\end{lstlisting}

Since \texttt{ring\_theory} was just built as
\texttt{\{\_,\_,...,\_\}} (syntax sugar for
insert, insert, ..., singleton), it suffices just to iteratively
try a couple of lemmas for membership of such a construction.

\begin{lstlisting}
lemma add_comm (a b : M) : a + b = b + a :=
begin
  /- M ⊨ ring_theory -> M ⊨ add_comm -/
  have hId : M ⊨ ring_signature.add_comm := h ring_signature.add_comm_in_ring_theory,
  /- M ⊨ add_comm -> add_comm b a -/
  have hab := hId b a,
  simpa [hab]
end\end{lstlisting}

There is some definitional and internal simplification happening in here,
but like before, for the most part \texttt{lean} recogizes that
realizing the sentence \texttt{add\_comm} is the same as having
an instance of \texttt{add\_comm}.

\begin{lstlisting}
instance comm_ring : comm_ring M :=
{
  add            := add,
  add_assoc      := add_assoc h,
  zero           := zero,
  zero_add       := zero_add h,
  add_zero       := add_zero h,
  neg            := neg,
  add_left_neg   := left_neg h,
  add_comm       := add_comm h,
  mul            := mul,
  mul_assoc      := mul_assoc h,
  one            := one,
  one_mul        := one_mul h,
  mul_one        := mul_one h,
  left_distrib   := mul_add h,
  right_distrib  := add_mul h,
  mul_comm       := mul_comm h,
}\end{lstlisting}

\subsection{Fields}

Our characterization of fields resembles the structure \texttt{is\_field}
more than the default \texttt{field} instance;
they are equivalent.

\begin{lstlisting}
structure is_field (R : Type u) [ring R] : Prop :=
(exists_pair_ne : ∃ (x y : R), x ≠ y)
(mul_comm : ∀ (x y : R), x * y = y * x)
(mul_inv_cancel : ∀ {a : R}, a ≠ 0 → ∃ b, a * b = 1)\end{lstlisting}

The proof that any field forms a model of the theory of fields is straight forward:
since fields are commutative rings, it is a model of \texttt{ring\_theory}
by our previous work; for the other two sentences we exploit \texttt{simp}
and all the lemmas about fields that already exist in \texttt{mathlib}.

\begin{lstlisting}
lemma realize_field_theory :
  Structure K ⊨ field_theory :=
begin
  intros ϕ h,
  cases h,
  {apply (comm_ring_to_model.realize_ring_theory K h)},
  repeat {cases h},
   { intro,
     simp only [fol.bd_or, models_ring_theory_to_comm_ring.realize_one,
       struc_to_ring_struc.func_map, fin.val_zero', realize_bounded_formula_not,
       struc_to_ring_struc.binaries_map, fin.val_eq_coe, dvector.last,
       realize_bounded_formula_ex, realize_bounded_term_bd_app,
       realize_bounded_formula, realize_bounded_term,
       fin.val_one, dvector.nth, models_ring_theory_to_comm_ring.realize_zero],
     apply is_field.mul_inv_cancel (K_is_field K) },
  { simp [fol.realize_sentence] },
  end\end{lstlisting}

Going backwards is even easier, and we omit the code here.

\subsection{Algebraically closed fields}

Suppose we have an algebraically field \texttt{K}.
We want to show that it is a model of the theory of algebraically closed fields,
which given our work so far amounts to showing that for each natural number $n$
we have that all generic monic polynomials of degree $n$ have a root in \texttt{k}.
Indeed using \texttt{is\_alg\_closed} we can obtain such a root for any polynomial,
but this requires (internally) making a polynomial corresponding \texttt{gen\_monic\_poly n}.
We first assume the existence of such a polynomial and that evaluating such a polynomial
at some value \texttt{x} is the same thing as realising \texttt{gen\_monic\_poly n}
at (its coefficients and then) \texttt{x}.

\begin{lstlisting}
/-- Algebraically closed fields model the theory ACF-/
lemma realize_ACF : Structure K ⊨ ACF :=
begin
  intros ϕ h,
  cases h,
  /- we have shown that K models field_theory -/
  { apply field_to.realize_field_theory _ h },
  { cases h with n hϕ,
    rw ← hϕ,
    /- goal is now to show that all generic monic polynomials of degree n have a root -/
    simp only [all_gen_monic_poly_has_root, realize_sentence_bd_alls,
      realize_bounded_formula_ex, realize_bounded_formula,
      models_ring_theory_to_comm_ring.realize_zero],
    intro as,
    have root := is_alg_closed.exists_root
      (polynomial.term_evaluated_at_coeffs as (gen_monic_poly n)) gen_monic_poly_non_const,
    cases root with x hx,
    rw polynomial.eval_term_evaluated_at_coeffs_eq_realize_bounded_term at hx,
    exact ⟨ x , hx ⟩ },
  end\end{lstlisting}

% check out the polynomial.term_evaluated_at_coeffs stuff and see if we can shorten
% the proof for models ACF -> alg_closed



\subsection{Characteristic}


\section{Internal completeness and soundness
  for Ax-Grothendieck}
It is a basic fact of linear algebra that any linear map
between vector spaces of the same finite dimension is
injective if and only if it is surjective.
\linkto{ax_groth_thm}{Ax-Grothendieck} says that this is
partly true for polynomial maps.

\begin{dfn}[Polynomial maps]
  \link{dfn_poly_map}
  Let $K$ be a commutative ring and $n$ a natural
  (we use $K$ since we are only interested in the case
  when it is an algebraically closed field).
  Let $f : K^n \to K^n$ such that for each $a \in K^n$,
  \[f(a) = (f_1(a), \dots, f_n(a))\] for
  $f_1, \dots, f_n \in K[x_1, \dots, x_n]$.
  Then we call $f$ a polynomial map over $K$.

  For the sake of computation it is simpler to simply assert
  the data of the $n$ polynomials directly:

  \begin{lstlisting}
  def poly_map (K : Type*) [comm_semiring K] (n : ℕ) : Type* :=
  fin n → mv_polynomial (fin n) K \end{lstlisting}

  Then we can take the map on types/sets by evaluating each polynomial
  \begin{lstlisting}
  def eval : poly_map K n → (fin n → K) → (fin n → K) :=
  λ ps as k, mv_polynomial.eval as (ps k) \end{lstlisting}
\end{dfn}

\begin{prop}[Ax-Grothendieck]
    \link{ax_groth_thm}
    Any injective polynomial map over an algebraically closed field is surjective.
    In particular injective polynomial maps over $\C$ are surjective.

\begin{lstlisting}
  theorem Ax_Groth {n : ℕ} {ps : poly_map K n} (hinj : function.injective (poly_map.eval ps)) :
  function.surjective (poly_map.eval ps) := sorry \end{lstlisting}
\end{prop}

The key lemma to prove this is the \linkto{lefschetz}{Lefschetz principle},
which says that ring theoretic statements are true in instances of algebraically closed fields
if and only if they are true in all algebraically closed fields
(assuming zero or large enough prime characteristic).
Lefschetz will be stated and proven in a later section.

An overview of the proof of Ax-Grothendieck follows:

\begin{itemize}
  \item We want to reduce the statement of Ax-Grothendieck to a model-theoretic one.
        Then we can apply the \linkto{lefschetz}{Lefschetz principle} to
        reduce to the prime characteristic case.
  \item To express ``for any polynomial map ...'' model-theoretically,
        which amounts to somehow quantifying over all polynomials
        in $n$ variables,
        we bound the degrees of all the polynomials,
        i.e. asking instead ``for any polynomial map consisting of
        polynomials with degree at most $d$''.
        Then we can write the polynomial as a sum of its monomials,
        with the coefficients as bounded variables.
  \item We express injectivity and surjectivity model-theoretically,
        and prove internal completeness and soundness for these statements.
  \item We apply Lefschetz, so that it suffices to prove Ax-Grothendieck for
        algebraic closures of a finite fields.
\end{itemize}

\subsection{Stating Ax-Grothendieck model-theoretically}

Our first objective is to state Ax-Grothendieck model-theoretically.
Let us assume we have an $n$-variable polynomial $p \in K[x_{1},\dots,x_{n}]$.
We know that $p$ can be written as a sum of its monomials,
and the set of monomials \texttt{monom\_deg\_le n d} is finite,
depending on the degree $d$ of the polynomial $p$.
It can be indexed by

\[ \texttt{monom\_deg\_le\_finset n d} := \set{ f : \texttt{fin n} \to \N \st \sum_{i < n} f i \le d }\]

Then we write
\[ p = \sum_{f\texttt{ : monom\_deg\_le n d}} p_{f}\prod_{i < n} x^{f i}\]

The typical approach to writing a sum like this in lean would be
to tell lean that only finitely many of the $p_{f}$ are non-zero
($p_{\star}$ is finitely supported - \texttt{finsupp}).
However, the API built for this assumes that the underlying
type in which the sum takes place is a commutative monoid,
which is not the case here,
as we will be expressing the above as a sum of terms
in the language of rings.
This type has addition and multiplication and so on,
which we supplied as \linkto{lean_symbols_for_ring_symbols}{instances} already,
but these are neither commutative nor associative.
Thus the workaround here was to use \texttt{list.sumr}
(my own definition, similar to \texttt{list.sum}) instead,
which will take a list of terms in the language of rings, and sum them together.

The below definition is meant to (re)construct polynomials as described above,
using free variables to represent the coefficients of some polynomial.
This can then be used to express injectivity and surjectivity.

\begin{lstlisting}
def poly_indexed_by_monoms (n d s p c : ℕ)
  (hndsc : (monom_deg_le n d).length + s ≤ c)
  (hnpc : n + p ≤ c) :
  bounded_ring_term c :=
list.sumr
(list.map
  (
    λ f : (fin n → ℕ),
    let
      x_js : bounded_ring_term c :=
      x_ ⟨((monom_deg_le n d).index_of' f + s) , ... ⟩,
      x_ip (i : fin n) : bounded_ring_term c :=
      x_ ⟨ (i : ℕ) + p , ... ⟩
    in
    x_js * (n.non_comm_prod (λ i, npow_rec (f i) (x_ip i)))
  )
  (monom_deg_le n d)
) \end{lstlisting}

To explain the above, we wish to express the ring term with $c$ many free variables
(``in context $c$'')
\[\sum_{f \in \texttt{monom\_deg\_le n d}} x_{j+s} \prod_{0 \le i < n} x_{i+p}^{f(i)}\]
\begin{itemize}
  \item When we write \texttt{x\_ < n , ... >} we are giving a natural $n$
        and a proof that $n$ is less than the context/variable bound $c$,
        which we omit here.
  \item \texttt{list.map} takes the list \texttt{monom\_deg\_le n d}
        (which is just \texttt{monom\_deg\_le\_finset n d} as a list instead\footnote{
          This uses the axiom of choice, in the form of \texttt{finset.to\_list}.})
        and gives us a list of terms looking like
        \[x_{js} \prod_{i < n} (x_{ip} i) ^{f i}\] one for each
        $f \in \texttt{monom\_deg\_le\_finset n d}$.
  \item Then \texttt{list.sumr} sums these terms together,
        producing a term in $c$ many free variables representing a polynomial.
  \item To define \texttt{x\_js} we take the index of $f$ in the list that $f$ came from
        and we add $s$ at the end and take the variable $x_{\texttt{index f} + s}$.
  \item To make the product we use \texttt{non\_comm\_prod}
        (this makes products indexed by \texttt{fin n},
        and works without commutativity or associativity conditions).
        For each $i < n$ we multiply together $x_{i + p}$.
  \item The purpose of adding $s$ and $p$ is to ensure we are not repeating variables in
        this expression. They give us control of where the variables begin and end.

        In the two situations where these polynomials are used $p$ is taken to be
        either $0$ or $n$; this makes the realizing variables $x_{0},\dots,x_{n - 1}$
        or $x_{n}, \dots, x_{2n - 1}$ represent evaluating the polynomials at values
        assigned to $x_{0},\dots,x_{2n - 1}$.

        The value $s$ in both instances taken to be
        $j \times \abs{\texttt{monom\_deg\_le\_finset n d}} + 2n$, where
        $j$ will represent the $j$-th polynomial
        (out of the $n$ polynomials from \texttt{poly\_map\_data}).
        This ensures that the variables between different polynomials
        in our polynomial map don't overlap.
\end{itemize}

\subsubsection*{Injectivity and surjectivity}

We can then express injectivity of a polynomial map.

\begin{lstlisting}
def inj_formula (n d : ℕ) :
  bounded_ring_formula (n * (monom_deg_le n d).length) :=
let monom := (monom_deg_le n d).length in
-- for all pairs in the domain x₋ ∈ Kⁿ and ...
bd_alls' n _
$
-- ... y₋ ∈ Kⁿ
bd_alls' n _
$
-- if at each pⱼ
(bd_big_and n
-- pⱼ x₋ = pⱼ y₋
  (λ j,
    (poly_indexed_by_monoms n d (j * monom + n + n) n _ _ ) -- note n
    ≃
    (poly_indexed_by_monoms n d (j * monom + n + n) 0 _ _ ) -- note 0
  )
)
-- then
⟹
-- at each 0 ≤ i < n,
(bd_big_and n ( λ i,
-- xᵢ = yᵢ (where yᵢ is written as xᵢ₊ₙ₊₁)
  x_ ⟨ i + n , ... ⟩ ≃ x_ (⟨ i , ... ⟩)
))
\end{lstlisting}

To explain the above, suppose we have $p$ the data of a polynomial map
(i.e. for each $j < n$ we have $p_j$ a polynomial).
We wish to express ``for all $x,y \in K^{n}$,
if $p x = p y$ then $x = y$''.
\begin{itemize}
  \item \texttt{bd\_alls' n} adds $n$ many $\forall$s in front of the
        formula coming after.
        The first represents $x = (x_{n},\dots,x_{2n-1})$ and the second represents
        $y = (y_{0},\dots,y_{n-1}) = (x_{0},\dots,x_{n-1})$.
        We choose this ordering since when we quantify this expression
        we first introduce $x$, which is of a higher index.
  \item \texttt{bd\_big\_and n} takes $n$ many formulas and places $\AND$s between
        each of them. In particular it expresses $p x = p y$, by breaking this up
        into the data of ``for each $j < n$, $p_{j} x = p_{j} y$'',
        as well as $x = y$, by breaking this up into the data of ``for each $i < n$,
        $x_{i+n} = x_{i}$''
  \item To write $p_{j} x$ and $p_{j} y$ we simply find the right variable indices
        to supply \texttt{poly\_indexed\_by\_monoms},
        and we ask for them to be equal.
\end{itemize}

Surjectivity is similar

\begin{lstlisting}
def surj_formula (n d : ℕ) :
  bounded_ring_formula (n * (monom_deg_le n d).length) :=
let monom := (monom_deg_le n d).length in
-- for all x₋ ∈ Kⁿ in the codomain
bd_alls' n _
$
-- there exists y₋ ∈ Kⁿ in the domain such that
bd_exs' n _
$
-- at each 0 ≤ j < n
bd_big_and n
-- pⱼ y₋ = xⱼ
λ j,
  poly_indexed_by_monoms n d (j * monom + n + n) 0 _
    inj_formula_aux0 inj_formula_aux1
  ≃
  x_ ⟨ j + n , ... ⟩ \end{lstlisting}

We wish to express ``for all $x \in K^{n}$, there exists $y \in K^{n}$ such that $p y = x$''.
Just like \texttt{bd\_alls' n}, \texttt{bd\_exs' n} adds $n$ many $\ex$s in front of the
formula coming after.

%% The statement of Ax-Groth in lean


\section{The Locally Finite Case}
Since Chris Hughes wrote the proof to this part of the project
I will only explain the mathematics behind the proof and not
talk about the \texttt{lean} formalization of it.

\begin{dfn}[Locally finite fields \cite{stack0}]
    \link{locally_finite}
    Let $K$ be a field of characteristic $p$ a prime.
    Then the following are equivalent definitions for $K$ being a
    \textit{locally finite field}:
    \begin{enumerate}
        \item The minimal subfield generated by any finite subset of $K$ is finite.
        \item $\F_p \to K$ is algebraic.
        \item $K$ embeds into an algebraic closure of $\F_p$.
    \end{enumerate}
    The important example for us of a locally finite field is an algebraic closure of $\F_{p}$.
    By the following theorem, this is a model of $\ACF_{p}$ satisfying Ax-Grothendieck.
\end{dfn}
\begin{proof}
  $1.\implies 2.$
  Let $a \in K$. Then $\F_{p}(a)$ is the minimal subfield generated by $a$,
  and is finite by assumption.
  Finite field extensions are algebraic $a$ is algebraic over $\F_{p}$.

  $2. \implies 1.$ We show by induction that $K$ is locally finite.
  Let $S$ be a finite subset of $K$.
  If $S$ is empty then $\F_p(S) = \F_p$ and so it is finite.
  If $S = T \cup {s}$ and $\F_p(T)$ is finite,
  then $s \in K$ is algebraic so by some basic field theory we can
  take the quotient by the minimal polynomial of $s$ giving
  \[\F_p(T)[x] / \min (s, \F_p(T)) \iso \F_p(S)\]
  The left hand side is finite because it is a finite
  dimensional vector space over a finite field.
  Hence $K$ is locally finite.

  $2. \iff 3.$ These are basic properties of algebraic closures.
\end{proof}

\begin{prop}[Ax-Grothendieck for locally finite fields]
  \link{ax_groth_locally_fin}
  Let $L$ be a locally finite field. Then any injective
  \linkto{dfn_poly_map}{polynomial map} $f : L^{n} \to L^{n}$ is surjective.
\end{prop}
\begin{proof}
  Let $b = (b_{1},\dots,b_{n}) \in L^{n}$.
  Writing $f = (f_1, \dots, f_n)$ for $f_i \in L[x_1, \dots, x_n]$
  we can find $A \subs L$,
  the set of all the coefficients of all of the $f_i$ when written out in monomials.
  $A \cup \set{b_{1},\dots,b_{n}}$ is finite and $L$ is locally finite,
  so the subfield $K$ generated by it is also finite.

  The restriction $\res{f}{K^n}$ is injective and has image inside $K^n$
  since each polynomial has coefficients in $K$ and is evaluated at
  an element of $K^n$.
  An injective endomorphism of a finite set is surjective,
  hence $b \in K^{n} = f(K^{n})$.
\end{proof}

% In fact the same idea can be used to prove the converse:
% that surjective polynomial maps on locally finite fields
% are injective.
%? Where does this stop being true??


\section{The Lefschetz Principle}
Returning to model theory of algebraically closed fields.
We begin by introducing the notion of a complete theory:

\begin{dfn}[Complete theories]
    \link{dfn_complete_theory}
    An $L$-theory $T$ is \textit{complete}
    when either of the following equivalent definitions hold:
    \begin{itemize}
      \item  $T$ deduces any sentence of its negation
    \begin{lstlisting}
      def is_complete' (T : Theory L) : Prop :=
      ∀ (ϕ : sentence L), T ⊨ ϕ ∨ T ⊨ ∼ ϕ \end{lstlisting}
      \item Sentences true in any model are deduced by the theory.
    \begin{lstlisting}
      def is_complete'' (T : Theory L) : Prop :=
      ∀ (M : Structure L) (hM : nonempty M) (ϕ : sentence L), M ⊨ T → M ⊨ ϕ → T ⊨ ϕ \end{lstlisting}
      \item All models of $T$ satisfy the same sentences
            (``are elementarily equivalent'').
    \end{itemize}
    Note that the definition \texttt{is\_complete} from the flypitch project
    is stronger than these conditions, and is useful when constructing
    theories with nice properties\footnote{
      Personally, I prefer the word maximal consistent theory for
      their definition \texttt{is\_complete}}.
    However in practice there is no reason to throw that many sentences
    into our language, so we use the versions above.
\end{dfn}
\begin{proof}
  The statement is
\begin{lstlisting}
  lemma is_complete''_iff_is_complete' {T : Theory L} :
    is_complete' T ↔ is_complete'' T := sorry \end{lstlisting}
  The forward direction involves casing on the hypothesis of $T \vDash \phi$
  or $T \vDash \neg \phi$, in the first case we are done,
  and in the second we get a contradiction by
  $\phi$ being both true and false in our model $M$.
\begin{lstlisting}
  { intros H M hM ϕ hMT hMϕ,
    cases H ϕ with hTϕ hTϕ,
    { exact hTϕ },
    {
      have hbot := hTϕ hM hMT,
      rw realize_sentence_not at hbot,
      exfalso,
      exact hbot hMϕ } },
\end{lstlisting}
    On the other hand we need to case on whether $T$
    is consistent or not.
    When $T$ is consistent we can show $T$ deduces
    $\phi$ or its negation by checking in that model,
    otherwise $T$ should deduce anything.
\begin{lstlisting}
  { intros H ϕ,
    by_cases hM : ∃ M : Structure L, nonempty M ∧ M ⊨ T,
    {
      rcases hM with ⟨ M , hM0 , hMT ⟩,
      by_cases hMϕ : M ⊨ ϕ,
      { left, exact H M hM0 ϕ hMT hMϕ },
      {
        right,
        rw ← realize_sentence_not at hMϕ,
        exact H M hM0 _ hMT hMϕ} },
    { left,
      intros M hM0 hMT,
      exfalso,
      apply hM ⟨ M , hM0 , hMT ⟩} } \end{lstlisting}
\end{proof}

\begin{prop}[Lefschetz principle]
    \link{lefschetz}
    Let $\phi$ be a sentence in the language of rings.
    Then the following are equivalent:
    \begin{enumerate}
        \item Some model of $\ACF_0$ satisfies $\phi$.
        (If you like $\C \vDash \phi$.)
        \item $\ACF_0 \vDash \phi$
        \item There exists $n \in \N$ such that for any prime $p$
            greater than $n$, $\ACF_p \vDash \phi$
        \item There exists $n \in \N$ such that for any prime $p$
        greater than $n$, some model of $\ACF_p$ satisfies $\phi$.
    \end{enumerate}
    The first and last equivalences are due to the theories $\ACF_{p}$
    being complete for any $p$ ($0$ or prime).
\end{prop}

To prove the above we need the following
\begin{itemize}
  \item \link{vaught_test}{Vaught's test} for showing a theory is complete
        (this does the first and last equivalences and
        is needed in the middle equivalence)
  \item The compactness theorem for the middle equivalence.
\end{itemize}
In this section we will introduce these notions properly and how
they are used.
Vaught's test will be proven in a \linkto{vaught_proof}{later section}.
The compactness theorem will not be proven
(it was part of the flypitch project).

\subsection{$\ACF_{n}$ is complete (Vaught's test)}
We want to show that $\ACF_{n}$ is complete.
Another way of expressing that a theory $T$ is complete is to
ask for models of $T$ to satisfy the same sentences
(that they are elementarily equivalent).
In particular it is known that isomorphic models %? Ref?
satisfy the same sentences.

\begin{dfn}[Categoricity]
    Given a language $L$ and a cardinal $\ka$,
    an $L$-theory $T$ is called $\ka$-categorical
    if any two models of $T$ of size $\ka$ are isomorphic.

    \begin{lstlisting}
  def categorical (κ : cardinal) (T : Theory L) :=
  ∀ (M N : Structure L) (hM : M ⊨ T) (hN : N ⊨ T), #M = κ → #N = κ → nonempty (M ≃[L] N) \end{lstlisting}
\end{dfn}

Vaught's test says that categoricity is a useful condition for showing a theory is complete.
We check that this holds for $\ACF_{n}$ and uncountable cardinals.
\begin{prop}[Categoricity for $\ACF_{n}$]
  If two algebraically closed fields have the same \textit{uncountable}
  cardinality then they are (non-canonically) isomorphic.
  \begin{lstlisting}
lemma ring_equiv_of_cardinal_eq_of_char_eq
  {K L : Type u} [hKf : field K] [hLf : field L]
  (hKalg : is_alg_closed K) (hLalg : is_alg_closed L)
  (p : ℕ) [char_p K p] [char_p L p]
  (hKω : ω < #K) (hKL : #K = #L) : nonempty (K ≃+* L) := sorry \end{lstlisting}

  Hence $\ACF_{n}$ is $\kappa$-categorical for any uncountable cardinal $\kappa$.
\end{prop}
\begin{proof}
  This is proven by Chris Hughes and is now part of mathlib.
  We outline the argument:

  Let $\F$ be the minimal field in $K$ and $L$,
  which is either finite or $\Q$
  and is the same field since they are of the same characteristic.

  There exist transcendence bases for $K$ and $L$ respectively,
  which we can call $s$ and $t$.
  Since $K$ and $L$ are both uncountable,
  the transcendence bases must be of the same cardinality as the fields.
  \[ \# K = \om + \# s = \# s \quad \text{ and }
    \quad \# L = \om + \# t = \# t \]
  Then $t$ and $s$ biject, hence we have ring isomorphisms
  \[K \iso \F(s) \iso \F(t) \iso L\]

  Then we can apply this to show categoricity:

\begin{lstlisting}
lemma categorical_ACF₀ {κ} (hκ : ω < κ) : fol.categorical κ ACF₀ :=
begin
  intros M N hM hN hMκ hNκ,
  haveI : fact (M ⊨ ACF₀) := ⟨ hM ⟩, haveI : fact (N ⊨ ACF₀) := ⟨ hN ⟩,
  split,
  apply equiv_of_ring_equiv,
  apply classical.choice,
  apply ring_equiv_of_cardinal_eq_of_char_zero, -- the char 0 version of what we showed above
  repeat { apply_instance },
  repeat { cc },
end\end{lstlisting}
\end{proof}

Another condition we will need for Vaught's test
is that there are only infinite models to the theory

\begin{lstlisting}
  def only_infinite (T : Theory L) : Prop := ∀ (M : Model T), infinite M.1\end{lstlisting}

This will hold in our case since algebraically closed fields are infinite.
We are now in a position to state Vaught's test.

\begin{prop}[Vaught's Test]
  \link{vaught_test}
  Let $L$ be a language and $T$ be a consistent theory in the language $L$
  with only infinite models, such that it is $\ka$-categorical
  for some large enough cardinal $\ka$ (see below for details).
  Then $T$ is a complete theory.

  \begin{lstlisting}
  lemma is_complete'_of_only_infinite_of_categorical
    [is_algebraic L] {T : Theory L} (M : Structure L) (hM : M ⊨ T)
    (hinf : only_infinite T) {κ : cardinal}
    (hκ : ∀ n, #(L.functions n) ≤ κ) (hωκ : ω ≤ κ) (hcat : categorical κ T) :
    is_complete' T := sorry
\end{lstlisting}
  This may differ slightly to the statement in other sources;
  the reason for the choice of these (stronger than usual)
  hypotheses will be discussed
  in the \linkto{vaught_proof}{section dedicated it its proof}.
  % In practice it is less work to prove this when $L$ has no relation symbols,
  % which is the case we are interested in (we say $L$ \textit{is algebraic}).
  % Another practical simplification is asking for $\kappa$ to be larger than
  % the collection of all function symbols.
  % The proof that Marker gives %? Ref
  % only requires that $\kappa$ is larger than the collection of constant symbols.
\end{prop}

We apply Vaught's test in our case to show that the theory
of algebraically closed fields of a fixed characteristic is complete.
However, before we do so we need a field theory lemma.


\begin{prop}
  $\ACF_{0}$ is complete and for any prime $p$, $\ACF_{p}$ is complete.
\end{prop}
\begin{proof}
  The two proofs are similar, so we focus on the characteristic $0$ case.
  According to Vaught's test, we first need to show that $\ACF_{0}$ is consistent,
  which we can do my simply giving a model: the algebraic closure of $\Q$.
  (For $\ACF_{p}$ we take the algebraic closure of $\F_{p}$.)
  We already have all the tools to make such a model:
  \begin{itemize}
    \item Mathlib has definitions of the rationals \texttt{rat} and finite fields \texttt{zmod}.
    \item (I lift them to an arbitrary universe level for generality.)
    \item Mathlib already has a definition of algebraic closure \texttt{algebraic\_closure}.
    \item We showed that any algebraically closed field is a model of $\ACF$
          and that characteristic $n$ fields are models of $\ACF_{n}$.
  \end{itemize}
  \begin{lstlisting}
def algebraic_closure_of_rat :
  Structure ring_signature :=
Rings.struc_to_ring_struc.Structure algebraic_closure.of_ulift_rat

instance algebraic_closure_of_rat_models_ACF : fact (algebraic_closure_of_rat ⊨ ACF) :=
by {split, classical, apply is_alg_closed_to.realize_ACF }

instance : char_zero algebraic_closure_of_rat := ...

theorem algebraic_closure_of_rat_models_ACF₀ :
  algebraic_closure_of_rat ⊨ ACF₀ :=
models_ACF₀_iff.2 ring_char.eq_zero \end{lstlisting}

The next thing to show is that any model of $\ACF_{0}$ is infinite.
This is true since any algebraically closed field is infinite
(I give a proof of this in \texttt{Rings.ToMathlib.algebraic\_closure};
it is just considering the roots of the separable polynomial $x^{n} - 1$ for each $0 < n$):
\begin{lstlisting}
lemma only_infinite_ACF : only_infinite ACF :=
  by { intro M, haveI : fact (M.1 ⊨ ACF) := ⟨ M.2 ⟩, exact is_alg_closed.infinite }\end{lstlisting}

We need a large cardinal for categoricity.
We choose this to be the continuum $\f{c}$, the cardinality of $\C$.
It is large enough since for each natural there are only finitely many function symbols
of that arity in the language of rings, and of course $\om \le \f{c}$.

Putting the above together we have
\begin{lstlisting}
theorem is_complete'_ACF₀ : is_complete' ACF₀ :=
is_complete'_of_only_infinite_of_categorical
    instances.algebraic_closure_of_rat
    instances.algebraic_closure_of_rat_models_ACF₀ -- algebraic closure of ℚ is a model of ACF₀
    (only_infinite_subset ACF_subset_ACF₀ only_infinite_ACF) -- alg closed fields are infinite
    -- pick the cardinal κ := 𝔠
    card_functions_omega_le_continuum
    omega_le_continuum
    (categorical_ACF₀ omega_lt_continuum) \end{lstlisting}
\end{proof}

\subsection{Compactness}

One way of stating compactness is the idea that proofs are finite.

\begin{prop}[Compactness (in terms of deduction)]
If $T$ is an $L$-theory and $f$ is an $L$-sentence
then $T$ deduces $\phi$ if and only if there is some finite subtheory of $T$
that deduces $f$.

\begin{lstlisting}
theorem compactness {L : Language} {T : Theory L} {f : sentence L} :
  T ⊨ f ↔ ∃ fs : finset (sentence L), (↑fs : Theory L) ⊨ (f : sentence L) ∧ ↑fs ⊆ T := sorry
\end{lstlisting}
\end{prop}

Confusingly, this can be found in a file called \texttt{completeness.lean}.
The backwards direction of this is obvious since any model of $T$ automatically
is a model of a finite subset.

There is an alternative formulation of compactness which we do not use for Lefschetz,
but is important as a tool for showing that a theory is consistent.
The reader may choose to come back to it when it is referred to later on.

\begin{prop}[Compactness (in terms of consistency)]
  If $T$ is an $L$-theory then $T$ is consistent if and only if
  each finite subtheory of $T$ is consistent.

  \begin{lstlisting}
theorem compactness' {L} {T : Theory L} : is_consistent T ↔
  ∀ fs : finset (sentence L), ↑fs ⊆ T → is_consistent (↑fs : Theory L) := \end{lstlisting}

  Often the term ``finitely consistent'' is used to describe the latter case.
\end{prop}

I prove the second statement (using the lemmas made for the first) in
\texttt{Rings.ToMathlib.completeness.lean}.
The proof I give below is \textit{not exactly} this proof,
since I wish to avoid first order logic syntax ($\vdash$),
which is the default layer of
definitions used in the \texttt{flypitch} library,
and just argue using model theory ($\vDash$).
However the essence of the proof is the same.

\begin{prop}
  The two formulations of compactness are equivalent.
\end{prop}
\begin{proof}
  \begin{forward}
    Clearly if $T$ is consistent with a model $\MM$ then $\MM$
    is also a model of any subtheory of $T$.

    For the converse we prove the contrapositive.
    Suppose $T$ is inconsistent, then $T \vDash \bot$,
    since proving this requires assuming a model of $T$.
    The first version of compactness
    implies there is a finite subset of $T$ that deduces $\bot$.
    This subset cannot be consistent, as any model will satisfy $\bot$.
  \end{forward}

  \begin{backward}
    Clearly if a finite subtheory of $T$ deduces a sentence $\phi$ then
    any model of $T$ is a model of the subtheory, hence also satisfies $\phi$.

    For the converse we again prove the contrapositive.
    \textit{Note that for a theory $\De$ and a sentence $\phi$ we have
      $\De \nvDash \phi$ if and only if $\De \cup \set{\neg \phi}$ is consistent.}
    We make use of this fact:
    Suppose all finite subtheories of $T$ do not deduce a sentence $\phi$.
    Then for any finite subtheory $\De \subs T$, we have $\De \nvDash \phi$ and so
    $\De \cup {\neg \phi}$ is consistent.
    Then $T \cup \set{\neg \phi}$ is a finitely consistent theory,
    hence is consistent by the second version of compactness.
    Hence $T \nvDash \phi$.
  \end{backward}
\end{proof}

\subsection{Proving Lefschetz}

We are now ready to prove the Lefschetz principle.
We begin by showing that if $\ACF_{0}$ deduces a ring sentence $\phi$
then $\ACF_{p}$ deduces $\phi$ for large $p$.
We prove it seperately because it will be used in the converse!

\begin{lstlisting}
theorem characteristic_change_left (ϕ : sentence ring_signature) :
ACF₀ ⊨ ϕ → ∃ (n : ℕ), ∀ {p : ℕ} (hp : nat.prime p), n < p → ACFₚ hp ⊨ ϕ := sorry \end{lstlisting}

\begin{proof}
We apply compactness:
if $\ACF_{0}$ deduces $\phi$ then we must have a finite subtheory of
$\ACF_{0}$ that deduces $\phi$.
In particular since $\ACF_{0}$ consisted of the axioms for $\ACF$ plus
$p + 1 \ne 0$ for each $p \in \N$ we know that only finitely many
such formulas are needed to deduce $\phi$.
Hence our $n$ should be the maximum $p$ such that $p + 1 \ne 0$
is in our finite subtheory, plus $1$.
\begin{lstlisting}
begin
  rw compactness,
  intro hsatis,
  obtain ⟨ fs , hsatis , hsub ⟩ := hsatis,
  classical,
  obtain ⟨ fsACF , fsrange , hunion, hACF , hrange ⟩ :=
    finset.subset_union_elim hsub,
  set fsnat : finset ℕ := finset.preimage fsrange plus_one_ne_zero
      (set.inj_on_of_injective injective_plus_one_ne_zero _) with hfsnat,
  use fsnat.sup id + 1, \end{lstlisting}

In the above \texttt{fs} is our finite subtheory,
\texttt{fsrange} is the part consisting of the formulas $p + 1 \ne 0$,
and the other part from $\ACF$ is called \texttt{fsACF}.

Let us then suppose that we have a prime that is larger than this $n$.
By design,
it should be that $\ACF_{p}$ deduces all the formulas from
our finite subtheory, hence $\ACF_{p}$ should deduce $\phi$.
Supposing that $M$ is a model of $\ACF_{p}$,
it suffices that $M$ deduces $\phi$.
Since $\texttt{fs} = \texttt{fsACF} \cup \texttt{fsrange}$
deduces $\phi$ it suffices that $M$ deduces \texttt{fsrange}
(it deduces $\ACF$ so it deduces any subset of $\ACF$).

\begin{lstlisting}
  intros p hp hlt M hMx hmodel,
  haveI : fact (M ⊨ ACF) := ⟨ (models_ACFₚ_iff'.mp hmodel).2 ⟩,
  have hchar := (@models_ACFₚ_iff _ _ _inst_1 _).1 hmodel,
  apply hsatis hMx,
  rw [← hunion, finset.coe_union, all_realize_sentence_union],
  split,
  { apply all_realize_sentence_of_subset _ hACF,
    exact all_realize_sentence_of_subset hmodel ACF_subset_ACFₚ},
  {sorry}, \end{lstlisting}

It suffices to show that for each $q$ in \texttt{fsnat}
(the list of $q$ such that $q + 1 \ne 0$ is in the subtheory \texttt{fs}),
$M$ satisfies $q + 1 \ne 0$.
We can then conclude that this is true since $M$ is characteristic $p$,
and $q < p$.
\end{proof}

Now for the whole theorem:

\begin{lstlisting}
theorem characteristic_change (ϕ : sentence ring_signature) :
ACF₀ ⊨ ϕ ↔ (∃ (n : ℕ), ∀ {p : ℕ} (hp : nat.prime p), n < p → ACFₚ hp ⊨ ϕ) := sorry \end{lstlisting}
\begin{proof}
It remains to prove the converse.
We know that $\ACF_{0}$ is complete,
so either $\ACF_{0} \vDash \phi$ or $\ACF_{0} \vDash \neg \phi$,
and it suffices to refute the latter case.

\begin{lstlisting}
begin
  split,
  { apply characteristic_change_left },
  {
    intro hn,
    cases is_complete'_ACF₀ ϕ with hsatis hsatis,
    { exact hsatis },
    { sorry }, \end{lstlisting}

We can apply the forward direction of Lefschetz,
and have that $\ACF_{p}$ deduces $\neg \phi$ for large $p$.
We instantiate these lower bounds, and take any prime $p$ that is larger
than their maximum.

\begin{lstlisting}
    { have hm := characteristic_change_left (∼ ϕ) hsatis,
      cases hn with n hn,
      cases hm with m hm,
      obtain ⟨ p , hle , hp ⟩ := nat.exists_infinite_primes (max n m).succ, \end{lstlisting}

We take the algebraic closure $\Om$ of $\F_{p}$ as a model of $\ACF_{p}$,
and since $p$ is suitably large, we have that $\Om \vDash \phi$ and
$\Om \vDash \neg \phi$, which is a contradiction.

\begin{lstlisting}
      have hnp : n < p :=
        lt_of_lt_of_le (nat.lt_succ_of_le (le_max_left _ _)) hle,
      have hmp : m < p :=
        lt_of_lt_of_le (nat.lt_succ_of_le (le_max_right _ _)) hle,
      have hS := instances.algebraic_closure_of_zmod_models_ACFₚ hp, -- Ω ⊨ ACFₚ
      specialize @hn p hp hnp _ ⟨ 0 ⟩ hS,
      specialize @hm p hp hmp _ ⟨ 0 ⟩ hS,
      simp only [realize_sentence_not] at hm,
      exfalso,
      apply hm hn } },
end \end{lstlisting}

\end{proof}


\section{Vaught's test and Upwards L\"{o}wenheim-Skolem}
In this section we go back to general model theory,
with the goal of proving \linkto{vaught_test}{Vaught's test}.
However, the proof of Vaught's test relies on a (a variant of) the
\linkto{upwards_lowenheim_skolem}{Upwards L\"{o}wenheim-Skolem Theorem}.
It says the following:

\begin{prop}[Upwards L\"{o}wenheim-Skolem]
  \link{upwards_lowenheim_skolem}
  Suppose $L$ is an algebraic language and $T$ is an $L$-theory.
  If $\kappa$ is a sufficiently large cardinal and $T$ has an infinite model,
  then $T$ has a model of size $\kappa$.

  \begin{lstlisting}
theorem has_sized_model_of_has_infinite_model [is_algebraic L] {T : Theory L} {κ : cardinal}
  (hκ : ∀ n, #(L.functions n) ≤ κ) (hωκ : ω ≤ κ) :
  (∃ M : Structure L, nonempty M ∧ M ⊨ T ∧ infinite M) →
  ∃ M : Structure L, nonempty M ∧ M ⊨ T ∧ #M = κ := sorry \end{lstlisting}
\end{prop}

This is often stated in terms of starting with an $L$-structure,
and extending it to a larger $L$-structure,
hence the word ``upward'' in the name.
This can be done using the above by taking $T$ to be the set of
sentences satisfied by the structure.

\subsection{Proof of Vaught's Test}
\link{vaught_proof}

We first apply \linkto{upwards_lowenheim_skolem}{
  Upwards L\"{o}wenheim-Skolem} to prove \linkto{vaught_test}{Vaught's test}.
Recall the statement:

\begin{lstlisting}
lemma is_complete'_of_only_infinite_of_categorical
  [is_algebraic L] {T : Theory L} (M : Structure L) (hM : M ⊨ T)
  (hinf : only_infinite T) {κ : cardinal}
  (hκ : ∀ n, #(L.functions n) ≤ κ) (hωκ : ω ≤ κ) (hcat : categorical κ T) :
  is_complete' T := sorry \end{lstlisting}

\begin{proof}
The proof is by contradiction.
Suppose $T$ is not complete;
this gives us a formula $\phi$ such that
\[ T \nvDash \phi \quad \text{and} \quad T \nvDash \neg \phi \]
which in turn (after unfolding the definition of $T \nvDash \phi$)
gives us two models $M$ and $N$ of $T$ such that
\[ M \nvDash \phi \quad \text{and} \quad N \nvDash \neg \phi \]
our aim is to adjust these to two models of $T$
of cardinality $\kappa$ so that they are isomorphic by categoricity,
but satisfy different sentences.

\begin{lstlisting}
begin
  intro ϕ,
  by_contra hbot,
  simp only [not_or_distrib, not_ssatisfied] at hbot,
  obtain ⟨ ⟨ M , hM0 , hM ⟩ , ⟨ N , hN0 , hN ⟩ ⟩ := hbot,
\end{lstlisting}

We can adjust cardinality using Upwards L\"{o}wenheim-Skolem,
obtaining models of cardinality $\kappa$.
This is why we need $T$ to only have infinite models.
\begin{lstlisting}
  obtain ⟨ M' , hM'0 , hM' , hMcard ⟩ := has_sized_model_of_has_infinite_model hκ hωκ
    ⟨
      M , hM0 , hM ,
      hinf ⟨ M , all_realize_sentence_of_subset hM (set.subset_insert _ _) ⟩
    ⟩,
  obtain ⟨ N' , hN'0 , hN' , hNcard ⟩ := has_sized_model_of_has_infinite_model hκ hωκ
    ⟨
      N , hN0 , hN ,
      hinf ⟨ N , all_realize_sentence_of_subset hN (set.subset_insert _ _) ⟩
    ⟩, \end{lstlisting}

By categoricity, $M$ and $N$ are isomorphic as $L$-structures.
We supply a proof that isomorphic structures satisfy the same
sentences in \texttt{Rings.ToMathlib.fol.lean}.
It follows from a series of proofs by induction on terms and formulas.

\begin{lstlisting}
  have hiso := hcat M' N'
    (all_realize_sentence_of_subset hM' (set.subset_insert _ _))
    (all_realize_sentence_of_subset hN' (set.subset_insert _ _)) hMcard hNcard,
  rw all_realize_sentence_insert at hM' hN',
  rw Language.equiv.realize_sentence _ (classical.choice hiso) at hN',
  exact hN'.1 hM'.1,
end
\end{lstlisting}
\end{proof}

\subsection{Upwards L\"{o}wenheim-Skolem}

Our remaining goal is to prove
\linkto{upwards_lowenheim_skolem}{Upwards L\"{o}wenheim-Skolem}.

\begin{lstlisting}
theorem has_sized_model_of_has_infinite_model [is_algebraic L] {T : Theory L} {κ : cardinal}
  (hκ : ∀ n, #(L.functions n) ≤ κ) (hωκ : ω ≤ κ) :
  (∃ M : Structure L, nonempty M ∧ M ⊨ T ∧ infinite M) →
  ∃ M : Structure L, nonempty M ∧ M ⊨ T ∧ #M = κ := sorry \end{lstlisting}

The idea of the proof is that we want to design a model of the right size
by making a language $L'$ extending $L$ and an $L'$-theory $T'$ extending $T$
(extending in the sense that any $L'$-model of $T'$ reduces down to a $L$-model of $T$),
such that the design of $L'$ and $T'$ guarantee that any model of $T'$ is large enough.
Meanwhile, we design the most obvious $L'$-model \texttt{term\_model} of $T'$,
by taking the type of all the $L'$-terms, and quotienting by equality deduced by $T'$,
guaranteeing that \texttt{term\_model} is small enough.


\subsection{Cardinality lemmas}

\subsubsection{Terms}

In this section we prove

\begin{lstlisting}
lemma bounded_preterm_le_functions {l} : #(bounded_preterm L n l) ≤
  max (cardinal.sum (λ n : ulift.{u} (ℕ), #(L.functions n.down))) ω := sorry \end{lstlisting}

There should be many approaches to this problem.
Mine was to note that preterms can be interpreted
the collection of all lists of presymbols that satisfy certain rules.
Then the list of all these symbols can be easily bounded above.
The preterm symbols can be made as an inductive type
\begin{lstlisting}
inductive preterm_symbol (L : Language) : Type u
| nat : ℕ → preterm_symbol
| var : Π {l}, fin l → preterm_symbol
| func : Π {l}, L.functions l → preterm_symbol
| app : preterm_symbol \end{lstlisting}

Then we inject any \texttt{bounded\_preterm L n l}
into the collection of lists of these preterm symbols.

\begin{lstlisting}
def preterm_symbol_of_preterm {n} : ∀ {l},
  bounded_preterm L n l → list (preterm_symbol L)
| _ (&k)         := [ preterm_symbol.var k ]
| l (bd_func f)  := [ preterm_symbol.func f ]
| l (bd_app t s) := [ preterm_symbol.app,
  preterm_symbol.nat (preterm_symbol_of_preterm t).length ]
  ++ preterm_symbol_of_preterm t ++ preterm_symbol_of_preterm s
 \end{lstlisting}

The choice of list as each image is designed to capture all
the pieces of data that went into constructing the term.
For example, if the term was built as a variable \texttt{\&k}
then we only need to include the data of how it was built
(\texttt{preterm\_symbol.var}), and that it used $k$,
so we take the list consisting of only
the preterm symbol \texttt{[ preterm\_symbol.var k ]}.
The case for a function symbol is similar.
More interestingly, when the preterm is built from
applying a preterm \texttt{t : bounded\_preterm L n (l + 1)}
to a preterm \texttt{s : bounded\_preterm L n 0},
we preserve the data of $t$ and $s$ by appending their
inductively given lists to the end of everything else we need.
It turns out that preserving the length of the list from $t$
is important for showing injectivity.

To show injectivity of the above we induct on $L$-terms $x$ and $y$.
There are $9$ cases to work on since there are $3$ cases for $x$ and $y$
respectively.

\begin{lstlisting}
lemma preterm_symbol_of_preterm_injective {l} :
  function.injective (@preterm_symbol_of_preterm L n l) :=
begin
  intros x,
  induction x with k _ _ _ tx sx htx hsx,
  { intro y,
    cases y,
    { intro h, simp only [...] at h, subst h },
    { intro h, cases h },
    { intro h, cases h } },
  { intro y,
    cases y,
    { intro h, cases h },
    { intro h, simp only [...] at h, subst h },
    { intro h, cases h } },
  { intro y,
    cases y with _ _ _ _ ty sy,
    { intro h, cases h },
    { intro h, cases h },
    { intro h, simp only [...] at h,
      obtain ⟨ ht , hs ⟩ := list.append_inj h.2 h.1,
      congr, { exact htx ht }, { exact hsx hs } } },
end \end{lstlisting}

The cases where $x$ and $y$ are not built by the same constructor
are easy to eliminate, since \texttt{no\_confusion}
for lists tells us two equal lists must have equal elements in the lists,
and \texttt{no\_confusion} for \texttt{preterm\_symbol} tells us
two equal preterm symbols must have come from the same constructor,
which yields a contradiction in each case.
This argument is hidden by the tactics \texttt{intro h, cases h},
where $h$ is the assumption that $x$ and $y$
make equal lists of preterm symbols.

The remaining cases: when both are variable symbols or both are function symbols
then we are assuming two lists with a single element are equal,
since the elements are the same constructor applied to some variable,
those variables must be equal by \texttt{no\_confusion} for
\texttt{preterm\_symbol}.
We thus have that $x = y$.
In the case when $x$ and $y$ are both applications,
we can use \texttt{no\_confusion} for lists and
apply injectivity of \texttt{list.append}
to deduce each part of the list is equal and apply the induction hypothesis.
Injectivity of \texttt{list.append} uses equality of lengths of the
sublists, which is why we included that data in our definition of
\texttt{preterm\_symbol\_of\_preterm}.

Now that we have an injection into \texttt{list (preterm\_symbol L)}
we should find the cardinality of \texttt{preterm\_symbol L},
which will determine the cardinality of lists of them.
We make a type equivalent to \texttt{preterm\_symbol L}:

\begin{lstlisting}
def preterm_symbol_equiv_fin_sum_formula_sum_nat :
  (preterm_symbol L) ≃
    (Σ l : ulift.{u} ℕ, ulift.{u} (fin l.down)) ⊕ (Σ l : ulift.{u} ℕ, L.functions l.down) ⊕ ℕ := ... \end{lstlisting}

This equivalence of types is obvious.
Equivalent types have the same cardinality, so
we can just compute the cardinality of the latter,
for which there is plenty of API.

Hence we can complete the lemma.
By the injection above we have the first inequality:
\begin{lstlisting}
lemma bounded_preterm_le_functions {l} : #(bounded_preterm L n l) ≤
  max (cardinal.sum (λ n : ulift.{u} (ℕ), #(L.functions n.down))) ω :=
calc #(bounded_preterm L n l) ≤ # (list (preterm_symbol L)) :
    cardinal.mk_le_of_injective (@preterm_symbol_of_preterm_injective L n l)
\end{lstlisting}
For an infinite type $\al$, $\# \al = \# \texttt{list } \al$.
Then replacing the cardinality along the equivalence above,
and going through some simple cardinal arithmetic proves the final inequality.
\begin{lstlisting}
  ... = # (preterm_symbol L) : cardinal.mk_list_eq_mk (preterm_symbol L)
  ... ≤ max (cardinal.sum (λ n : ulift.{u} (ℕ), #(L.functions n.down))) ω :
begin
  rw cardinal.mk_congr (preterm_symbol_equiv_fin_sum_formula_sum_nat L),
  simp only [...],
  apply le_trans (cardinal.add_le_max _ _) (max_le (max_le _ _) (le_max_right _ _)),
  { apply le_max_of_le_right,
    apply le_trans (cardinal.sum_le_sup.{u} (λ (i : ulift.{u} ℕ), (i.down : cardinal.{u}))),
    apply le_trans (cardinal.mul_le_max _ _) (max_le (max_le _ _) (le_of_eq rfl)),
    { simp },
    { rw cardinal.sup_le, intro i, apply le_of_lt, rw cardinal.lt_omega, simp, }
  },
  { apply le_trans (cardinal.add_le_max _ _) (max_le (max_le _ _) (le_max_right _ _)),
    { simp },
    { exact le_max_right _ _ } }
end \end{lstlisting}


\subsubsection{Formulas}

\subsubsection{Henkinization}

\bibliography{refs}{}
\bibliographystyle{abbrv}










\end{document}
